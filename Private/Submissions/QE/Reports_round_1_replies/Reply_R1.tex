\documentclass[12pt,letterpaper,english]{article}
\usepackage[round, authoryear]{natbib}
\usepackage{floatrow}
\renewcommand{\floatpagefraction}{.99}
\newfloatcommand{capbtabbox}{table}[][\FBwidth]
\usepackage{eurosym}
\usepackage{graphicx}
\usepackage{setspace}
\usepackage{geometry}
\usepackage{bbm}
\usepackage{hyperref}
\usepackage[titletoc]{appendix}
\usepackage{graphicx}
\newcommand{\argmin}{\arg\!\min}
\usepackage{amsmath, amssymb,amsthm,mathtools,dsfont}
\usepackage{algorithmic}
\usepackage{booktabs}
\usepackage{setspace}
\usepackage{enumerate}
\usepackage[flushleft]{threeparttable}
\usepackage{rotating}
\usepackage{float}
\usepackage{tabulary}
\usepackage{ragged2e}
\usepackage{rotating}
\usepackage{epstopdf}
\usepackage[labelfont=bf]{caption}
\usepackage{array}
\newcolumntype{C}[1]{>{\centering\let\newline\\\arraybackslash\hspace{0pt}}m{#1}}
\usepackage{subfig}
\usepackage{placeins}
\usepackage{pxfonts}
\usepackage{xcolor}
\usepackage{svg}
\usepackage{multirow}
\usepackage{verbatim}
\hypersetup{
	colorlinks,
	linkcolor={red!50!black},
	citecolor={red!50!black},
	urlcolor={blue!80!black}
}
\textheight=23cm \textwidth=16.5cm \oddsidemargin=0cm
\evensidemargin=0cm \topmargin=-1.75cm
\setcounter{MaxMatrixCols}{10}
\makeatletter
\g@addto@macro\@floatboxreset\centering
\makeatother
\setlength\floatsep{2\baselineskip plus 3pt minus 2pt}
\setlength\textfloatsep{2\baselineskip plus 3pt minus 2pt}
\setlength\intextsep{2\baselineskip plus 3pt minus 2 pt}
\newcommand*{\MyIndent}{\hspace*{0.5cm}}%
\usepackage[normalem]{ulem}
\newtheorem{theorem}{Proposition}
\newtheorem{definition}{Definition}
\newtheorem{proposition}{Proposition}
\newtheorem{corollary}{Corollary}
\newtheorem{example}{Example}
\newtheorem{lemma}{Lemma}
\newtheorem{remark}{Remark}
\newtheorem{assumption}{Assumptions}
\newtheorem{hypothesis}{Hypothesis}


% Uncomment the next line to use the harvard package with bibtex
%\usepackage[abbr]{harvard}

% This command determines the leading (vertical space between lines) in draft mode
% with 1.5 corresponding to "double" spacing.
%\draftSpacing{1.5}
%\newlength\TableWidth
\usepackage{array}
\newcolumntype{H}{>{\setbox0=\hbox\bgroup}c<{\egroup}@{}}
\newcommand{\Figures}{Figures/}
\newcommand{\Tables}{Tables/}
\usepackage{pifont}
\newcommand{\cmark}{\ding{51}}%
\newcommand{\xmark}{\ding{55}}%

\title{\textbf{Response to Referee 1\\ Quantitative Economics MS 2442 \\``Welfare and Spending Effects of \\ Consumption Stimulus Policies''}}
\author{Christopher D. Carroll, Edmund Crawley, William Du, \\ Ivan Frankovic, and H\aa kon Tretvoll}
\date{}

\begin{document}
	\onehalfspacing
	\maketitle
	
\noindent Thank you for your thoughtful comments and suggestions on our paper ``Welfare and Spending Effects of Consumption Stimulus Policies''. They were all very useful to us in revising the paper. We hope you agree that the paper has improved. In the following, we summarize the main changes we have made based on your, the other referees', and the editor's suggestions. Thereafter, we state each of your comments in italics and provide point-by-point responses to them.

\section{Summary of Main Changes}

\begin{enumerate}
	\item We have \ldots 
\end{enumerate}

\newpage 

\section{Essential points}
\begin{enumerate}
	\item \textit{\textbf{Partial equilibrium.} The main drawback of the analysis is the partial equilibrium nature. \citet{kekre2022unemp} and Broer et al (2023), in contrast, seem to find general-equilibrium effects to be important. I think the PE approach still gives interesting results, but the authors should be more up-front about it (the word "partial equilibrium" appears first on page 11), and discuss its shortcomings.}
	
	\paragraph{Response.} 
	
	\item \textit{\textbf{Tax policy.} Do taxes rise eventually to pay for the policies? I don't think so, but this is extremely unclear. ``Should'' appears four times in the discussion of the financing of government policies. The authors should introduce an explicit rule for tax policy (even if paid very far in the future), as this may matter for consumption of the Ricardian / high-wealth households.}
	
	\paragraph{Response.} 
	
	\item \textit{\textbf{`Splurge' and utility.} The implications of splurge consumption should be discussed more in detail. Take, e.g., the assumption that households gain utility only from post-splurge consumption. This implies that, essentially, the model is equivalent to an alternative where all incomes are taxed by 30 percent (and immediately go into government consumption). But doesn't this increase further, and mechanically, the effect of transfers to the low-wealth unemployed (who consume post-splurge income, increasing marginal utility by 1/0.7=1.42 relative to that from total income) relative to high-wealth households (who consume permanent income from financial wealth)?}
		
	\paragraph{Response.} 

	\item \textit{\textbf{Calibration of the `splurge'.} I find the calibration strategy, to estimate the `splurge' on Norwegian data (plus a US liquid-wealth distribution...), and then calibrate the rest to US data only, confusing - also since the authors then compare their final average annual MPC to an estimate for Norway. I would appreciate a clear motivation for why we need the Norwegian data targets at the beginning of Section 3.1 - I guess it's because we don't have estimates for MPCs by wealth for the US. An alternative, perhaps cleaner, calibration would be to US data only with some sensitivity analysis for alternative values of the splurge-fraction. In any case, I would appreciate a (perhaps only verbal) comparison to the more common `share-of-hand-to-mouth-agents' calibration (that doesn't capture wealthy high-MPC agents but may have similar demand properties).}
	
	\paragraph{Response.} 

\bigskip
\bigskip

\item \textit{\textbf{Additional comments}} 

	\begin{enumerate}
		\item \textit{\textbf{Clarification of the calibration.} I didn't understand if the model is for households, or individuals, and what data target pertains to which of the two (the text uses both words, although "household" dominates).}
		
		\paragraph{Response.} 
		
		\item \textit{\textbf{Additional policies.} To give the payroll tax cut a better chance, couldn't the authors consider a ``payroll tax cut as long as the recession lasts''? And I don't understand why the authors don't consider an increase in UI generosity, which also is a common policy.}
						
		\paragraph{Response.} 

		\item \textit{\textbf{Motivation of modeling choices.} Why the two-year horizon for the payroll-tax	reduction? Consumers expect this policy to continue with 50 percent	probability if the recession continues for longer than 2 years. Why?}

		\paragraph{Response.} 

		\item \textit{\textbf{Consumption drop upon unemployment.} The authors should report the drop in consumption upon unemployment benefit expiry, and compare it 	to the estimates in \citet{ganongConsumer2019}.}

		\paragraph{Response.} 
		
		\item \textit{\textbf{Why the quarterly calibration?} This implies that the unemployed are at least unemployed for three months, which is high relative to U.S. job-finding rates. The unemployment-rate targets should be adjusted for that.}

		\paragraph{Response.} 
		
		\item \textit{\textbf{Job-finding rates.} Why the homogeneous job-finding rates across education groups and duration? It seems unnecessary (since the model keeps track of individual duration and education) and probably produces counterfactually few low-skill long-term unemployed.}
		
		\paragraph{Response.} Our calibration strategy for the transitions into and out of unemployment is to match the data that we have from the Bureau of Labor Statistics. This includes average unemployment rates by education groups, and the average duration of unemployment spells, but it does not include data on education-specific average durations. As we are using the 2004 wave of the SCF, we use the average number of weeks unemployed from 2004 which was 19.6 weeks, almost exactly 1.5 quarters. This pins down the job-finding rate while unemployed for all education groups, and the education-specific transition probabilities from employment to unemployment are then calibrated to match the education-specific unemployment rates. 
		
		This calibration is also consistent with the results in \citet{mincer1991education} who finds that the main difference between education groups is in the incidence of unemployment, and not its duration. He states that ``the reduction	of the incidence of unemployment [at higher education levels] is found to be far more important than the reduced duration of unemployment in creating the educational differentials in unemployment rates'' (page 1). \citeauthor{mincer1991education} reports probabilities of transitioning into unemployment as the product of the probability of job separation times the conditional probability of unemployment given a job separation. Both of these are lower for higher education levels. For our calibration, this means that a higher job finding rate \textit{within} the quarter of the job separation for more educated workers translates into a lower probability of transitioning from employment to unemployment. 
		
		\citet{mincer1991education} refers to data from 1979 and 1980, and there may have been substantial changes in the labor market since then. However, more recent work by \citet{elsby2010labor} echoes his findings. Our calibration strategy is consistent both with these results and the data we have. 
		%``[the] duration of unemployment is a relatively minor aspect of the educational unemployment differentials'' (page 6)
		
	\end{enumerate}
\end{enumerate}

\bigskip

\noindent Finally, we would like to thank you again for your careful advice on our paper. We hope you find our revision satisfactory.

\bibliographystyle{econometrica}
\bibliography{../../../../HAFiscal.bib, ../../../../Add-Refs.bib}
\end{document}
