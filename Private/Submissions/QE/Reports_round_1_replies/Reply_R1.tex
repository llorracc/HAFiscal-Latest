\documentclass[12pt,letterpaper,english]{article}
\usepackage[round, authoryear]{natbib}
\usepackage{floatrow}
\renewcommand{\floatpagefraction}{.99}
\newfloatcommand{capbtabbox}{table}[][\FBwidth]
\usepackage{eurosym}
\usepackage{graphicx}
\usepackage{setspace}
\usepackage{geometry}
\usepackage{bbm}
\usepackage{hyperref}
\usepackage[titletoc]{appendix}
\usepackage{graphicx}
\newcommand{\argmin}{\arg\!\min}
\usepackage{amsmath, amssymb,amsthm,mathtools,dsfont}
\usepackage{algorithmic}
\usepackage{booktabs}
\usepackage{setspace}
\usepackage{enumerate}
\usepackage[flushleft]{threeparttable}
\usepackage{rotating}
\usepackage{float}
\usepackage{tabulary}
\usepackage{ragged2e}
\usepackage{rotating}
\usepackage{epstopdf}
\usepackage[labelfont=bf]{caption}
\usepackage{array}
\newcolumntype{C}[1]{>{\centering\let\newline\\\arraybackslash\hspace{0pt}}m{#1}}
\usepackage{subfig}
\usepackage{placeins}
\usepackage{pxfonts}
\usepackage{xcolor}
\usepackage{svg}
\usepackage{multirow}
\usepackage{verbatim}
\hypersetup{
	colorlinks,
	linkcolor={red!50!black},
	citecolor={red!50!black},
	urlcolor={blue!80!black}
}
\textheight=23cm \textwidth=16.5cm \oddsidemargin=0cm
\evensidemargin=0cm \topmargin=-1.75cm
\setcounter{MaxMatrixCols}{10}
\makeatletter
\g@addto@macro\@floatboxreset\centering
\makeatother
\setlength\floatsep{2\baselineskip plus 3pt minus 2pt}
\setlength\textfloatsep{2\baselineskip plus 3pt minus 2pt}
\setlength\intextsep{2\baselineskip plus 3pt minus 2 pt}
\newcommand*{\MyIndent}{\hspace*{0.5cm}}%
\usepackage[normalem]{ulem}
\newtheorem{theorem}{Proposition}
\newtheorem{definition}{Definition}
\newtheorem{proposition}{Proposition}
\newtheorem{corollary}{Corollary}
\newtheorem{example}{Example}
\newtheorem{lemma}{Lemma}
\newtheorem{remark}{Remark}
\newtheorem{assumption}{Assumptions}
\newtheorem{hypothesis}{Hypothesis}


% Uncomment the next line to use the harvard package with bibtex
%\usepackage[abbr]{harvard}

% This command determines the leading (vertical space between lines) in draft mode
% with 1.5 corresponding to "double" spacing.
%\draftSpacing{1.5}
%\newlength\TableWidth
\usepackage{array}
\newcolumntype{H}{>{\setbox0=\hbox\bgroup}c<{\egroup}@{}}
\newcommand{\Figures}{Figures/}
\newcommand{\Tables}{Tables/}
\usepackage{pifont}
\newcommand{\cmark}{\ding{51}}%
\newcommand{\xmark}{\ding{55}}%

\title{\textbf{Response to Referee 1\\ Quantitative Economics MS 2442 \\``Welfare and Spending Effects of \\ Consumption Stimulus Policies''}}
\author{Christopher D. Carroll, Edmund Crawley, William Du, \\ Ivan Frankovic, and H\aa kon Tretvoll}
\date{}

\begin{document}
	\onehalfspacing
	\maketitle
	
\noindent Thank you for your thoughtful comments and suggestions on our paper ``Welfare and Spending Effects of Consumption Stimulus Policies''. They were all very useful to us in revising the paper. We hope you agree that the paper has improved. In the following, we summarize the main changes we have made based on your, the other referees', and the editor's suggestions. Thereafter, we state each of your comments in italics and provide point-by-point responses to them.

\section{Summary of Main Changes}

\begin{enumerate}
	\item We have \ldots 
\end{enumerate}

\newpage 

\section{Essential points}
\begin{enumerate}
	\item \textit{\textbf{Partial equilibrium.} The main drawback of the analysis is the partial equilibrium nature. \citet{kekre2022unemp} and Broer et al (2023), in contrast, seem to find general-equilibrium effects to be important. I think the PE approach still gives interesting results, but the authors should be more up-front about it (the word "partial equilibrium" appears first on page 11), and discuss its shortcomings.}
	
	\noindent \textbf{Response.} 
	
	\item \textit{\textbf{Tax policy.} Do taxes rise eventually to pay for the policies? I don't think so, but this is extremely unclear. ``Should'' appears four times in the discussion of the financing of government policies. The authors should introduce an explicit rule for tax policy (even if paid very far in the future), as this may matter for consumption of the Ricardian / high-wealth households.}
	
	\noindent \textbf{Response.} 
	
	\item \textit{\textbf{`Splurge' and utility.} The implications of splurge consumption should be discussed more in detail. Take, e.g., the assumption that households gain utility only from post-splurge consumption. This implies that, essentially, the model is equivalent to an alternative where all incomes are taxed by 30 percent (and immediately go into government consumption). But doesn't this increase further, and mechanically, the effect of transfers to the low-wealth unemployed (who consume post-splurge income, increasing marginal utility by 1/0.7=1.42 relative to that from total income) relative to high-wealth households (who consume permanent income from financial wealth)?}
		
	\noindent \textbf{Response.} 

	\item \textit{\textbf{Calibration of the `splurge'.} I find the calibration strategy, to estimate the `splurge' on Norwegian data (plus a US liquid-wealth distribution...), and then calibrate the rest to US data only, confusing - also since the authors then compare their final average annual MPC to an estimate for Norway. I would appreciate a clear motivation for why we need the Norwegian data targets at the beginning of Section 3.1 - I guess it's because we don't have estimates for MPCs by wealth for the US. An alternative, perhaps cleaner, calibration would be to US data only with some sensitivity analysis for alternative values of the splurge-fraction. In any case, I would appreciate a (perhaps only verbal) comparison to the more common `share-of-hand-to-mouth-agents' calibration (that doesn't capture wealthy high-MPC agents but may have similar demand properties).}
	
	\noindent \textbf{Response.} 

\bigskip
\bigskip

\item \textit{\textbf{Additional comments}} 

	\begin{enumerate}
		\item \textit{\textbf{Clarification of the calibration.} I didn't understand if the model is for households, or individuals, and what data target pertains to which of the two (the text uses both words, although "household" dominates).}
		
		\noindent \textbf{Response.} Our model is a simplified model of households. This fits with our calibration to the distributions of liquid wealth from the SCF which is a survey of families in the US. It is simplified, however, in that we do not take into account heterogeneity across household size or composition. The classification of households into three education groups is based on the level of education of the head of the household. The income risk and employment/unemployment transitions can also be thought of as applying to the head of the household as the calibration targets from the Bureau of Labor statistics are for unemployment rates and duration for individuals. However, we take into account that the model is for a household in the calibration of the replacement rates with or without unemployment benefits. The parameters describing the unemployment system is based on \citet{rothstein2017scraping} who study the effects of unemployment on household income. 
		
		In response to your comment, we have clarified our presentation of the calibration. 
		
		\item \textit{\textbf{Additional policies.} To give the payroll tax cut a better chance, couldn't the authors consider a ``payroll tax cut as long as the recession lasts''? And I don't understand why the authors don't consider an increase in UI generosity, which also is a common policy.}
						
		\noindent \textbf{Response.} We do not consider a payroll tax cut that only applies during a recession as we view it as unlikely that policy makers would implement such a policy. The end date of a recession is only identified with a significant lag by the NBER Business Cycle Dating Committee. A tax cut that was a function of this retrospective dating of a recession by the committee would then imply a tax increase (reversal of the tax cut) with potentially several months of retroactive effect. In the model, the tax cut could, of course, expire as soon as the economy transitioned out of the recession, but this option does not seem available in practice. 
		
		One reason we have focused on the policy of extending instead of expanding UI benefits is that the policies we consider are motivated by policies implemented in the aftermath of the Great Recession, and the Tax Relief, Unemployment Insurance Reauthorization, and Job Creation Act of 2010 included an extension of federal unemployment benefits. Although an increase in UI generosity is also a policy that has been applied, for example the CARES act implemented during the COVID-19 pandemic included both extension and expansion of benefits, we view the extension of eligibility for benefits as the more common policy. 
		
		It is also worth noting the point made by \citet{ganongconsumer2019} that UI extensions is likely to be the more important policy. The considerable drop in spending upon the expiry of unemployment benefits that they identify leads them to conclude that ``the consumption-smoothing gains from extending UI benefits are four times greater than from increasing the level of UI benefits'' (page 2384).
		
		Our framework could be extended to also consider a policy of making UI benefits more generous. At the moment, our code is not set up to do that, however, and while changing the codes to include this policy would be relatively straightforward, it would be computationally costly to recompute all our results for an additional policy. 

		Finally, in a robustness section that we have included in this version, we consider some of the general equilibrium effects of stimulus policies in a HANK-SAM model. In that framework we also considered a policy of increasing UI generosity and found that \colorbox{yellow}{XX}. However, we have not included these results in the paper. 
	
		\item \textit{\textbf{Motivation of modeling choices.} Why the two-year horizon for the payroll-tax	reduction? Consumers expect this policy to continue with 50 percent	probability if the recession continues for longer than 2 years. Why?}

		\noindent \textbf{Response.} The two-year horizon for the payroll-tax reduction is inspired by the two-year tax cut implemented in the Tax Relief, Unemployment Insurance Reauthorization, and Job Creation Act of 2010. This tax cut was itself an extension of previously implemented cuts known as the ``Bush tax cuts''. In general, we view it as a significantly easier process for policy makers to extend a tax cut that is in place than to start the legislative process from scratch to implement a new policy. Therefore, we let consumers expect that the policy may continue if the recession outlasts the two-year horizon of the initial tax cut.
		
		In response to your questions, we have extended our discussion of the payroll tax cut policy to include these points. 

		\item \textit{\textbf{Consumption drop upon unemployment.} The authors should report the drop in consumption upon unemployment benefit expiry, and compare it to the estimates in \citet{ganongconsumer2019}.}

		\noindent \textbf{Response.} Thank you for this suggestion. We now include a discussion of the consumption drop upon unemployment benefit expiry in our model in section~\colorbox{yellow}{XX}. We have also added a reference to the results in \citet{ganongconsumer2019} when discussing the motivation for including splurge consumption in the model. 
	
		\item \textit{\textbf{Why the quarterly calibration?} This implies that the unemployed are at least unemployed for three months, which is high relative to U.S. job-finding rates. The unemployment-rate targets should be adjusted for that.}

		\noindent \textbf{Response.} We chose a quarterly calibration since this is the standard in the literature on business cycles and we explicitly consider policies in reponse to recessions. We consider it realistic that a recession starts and a policy is implemented within the same period when the period in the model is one quarter.
		
		We don't think the education-specific unemployment rate targets should be adjusted because of the frequency of the model, but we agree that the frequency is important for the interpretation of the probabilities of transitioning into and out of unemployment in our calibration. Further discussion of your comments here is incorporated in the reply to your point below concerning job-finding rates. 
		
		\item \textit{\textbf{Job-finding rates.} Why the homogeneous job-finding rates across education groups and duration? It seems unnecessary (since the model keeps track of individual duration and education) and probably produces counterfactually few low-skill long-term unemployed.}
		
		\noindent \textbf{Response.} Our calibration target for the probability of transitioning from unemployment to employment is the average duration of unemployment. As we are using the 2004 wave of the SCF, we use the average number of weeks unemployed from 2004 from the Bureau of Labor Statistics, which was 19.6 weeks = 4.5 months = 1.5 quarters. With a quarterly model we can match that average, and since we do not have data on education-specific durations of unemployment, we do so by calibrating the transition probability from unemployment to employment to be the same across education groups. With a monthly calibration we would match the same average with a lower transition probability, without the implication that those who become unemployed would remain unemployed for at least three months. 
		
		The education-specific element of our calibration is to target the average unemployment rates by education group. We do so by calibrating education-specific transition probabilities from employment to unemployment. Our calibration strategy is consistent with the results in \citet{mincer1991education} who finds that the main difference between education groups is in the incidence of unemployment, and not its duration. He states that ``the reduction of the incidence of unemployment [at higher education levels] is found to be far more important than the reduced duration of unemployment in creating the educational differentials in unemployment rates'' (page 1). 
		
		\citeauthor{mincer1991education} reports probabilities of transitioning into unemployment as the product of the probability of job separation times the conditional probability of unemployment given a job separation, and both of these are lower for higher education levels. For our calibration, this means that a higher job finding rate \textit{within} the quarter of the job separation for more educated workers translates into a lower probability of transitioning from employment to unemployment during a quarter. In that sense, our calibration is consistent with short-term job-finding rates being higher for more educated workers. 
		
		\citet{mincer1991education} refers to data from 1979 and 1980, and there may have been substantial changes in the labor market since then. However, more recent work by \citet{elsby2010labor} include data up to 2009 and echo \citeauthor{mincer1991education}'s findings (see their Figure 8). 
		
		In response to your questions and comments, we have updated our discussion of the calibration of these transition probabilities in the text and added the references discussed here. 
		
	\end{enumerate}
\end{enumerate}

\bigskip

\noindent Finally, we would like to thank you again for your careful advice on our paper. We hope you find our revision satisfactory.

\bibliographystyle{econometrica}
\bibliography{../../../../HAFiscal.bib}
\end{document}
