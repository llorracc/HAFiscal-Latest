\begin{itemize}
	\item \textbf{The Splurge} We have added section~4.5 and Appendix~A to discuss the implications of the splurge for our model. 
	
	To do so we have reestimated the US model without imposing any splurge consumption. We show that with a wider distribution of discount factors, the model is still able to match the empirically observed liquid wealth distribution in the US. It also matches untargeted moments on the consumption dynamics in response to transitory income shocks fairly well and still generates a substantial drop in consumption when unemployment benefits expire. The empirical fit is not as good as for the model that includes a splurge, however.  
	
	We then simulate the three fiscal policies in the reestimated models, and we show that our multiplier and welfare metrics for the policies differ only marginally between the model with and without the splurge. Importantly, the ranking of the policies remains unaffected. While the splurge is thus helpful in matching available empirical evidence, it does not affect the main conclusions of our paper. In fact, including the splurge in the model, increases the cumulative multipliers of both the stimulus check and the tax policy, while leading to a slight reduction in the multipliers for the UI extension. 
	
	The key question is whether it is necessary for our purposes to include splurge consumption in the model. Including the splurge provides a better fit to the dynamics of spending after a temporary income shock, and when ranking the policies we discuss the timing of spending that they induce as an important distinguishing characteristic. For this reason, we have opted to keep it in our baseline version of the model. However, the empirical fit without the splurge is not so much worse that it substantially affects our results. The main moment the model cannot generate without the splurge is the high MPC for the wealthiest quartile, which is not so important for our evaluation of consumption stimulus policies. 
	
	%For this reason we have also substantially rewritten our motivation in the introduction of the paper for why we include the splurge, pointing out that the splurge acts as a stand-in for competing theories for why agents may spend more out of transitory income shocks than suggested by a simple model in which agents solely maximize utility. We believe our finding that the splurge does not matter too greatly for consumption dynamics to affect our policy conclusions, is an interesting finding in itself. As the splurge still provides a slightly better fit of the empirical targets, we have opted to keep the version with the splurge as our baseline model.
	
	
	\item \textbf{Welfare Measure} We have completely overhauled our section on welfare and have introduced a new measure that we think best captures the idea that we want to measure welfare gains from carrying out each policy during a recession, but give no benefit in normal times to policies that in our model would increase welfare through redistribution. 
	
	Our welfare measure weights the felicity of a household at time $t$ by the inverse of the marginal utility of the same household in a counterfactual simulation in which neither the recession occurred nor the fiscal policy was implemented. This weighting scheme means that in normal times the marginal benefit to a social planner of moving a dollar of consumption from one household at one time period to another household at the same or different time period is zero. Hence, in normal times, any re-distributive policy has zero marginal benefit. However, in a recession when the average marginal utility is higher than in normal times, there can be welfare benefits to government borrowing to allow households to consume more during the recession.
	
	Our new welfare measure leads to the same qualitative conclusions as in the previous version of the paper. It improves on the previous measure in several ways:
	\begin{itemize}
		\item The new measure does not scale with the size of the fiscal stimulus. We divide by the net present value of the fiscal policy so the measure is a `bang-bang-for-the-buck' measure. As pointed out by referee 2, our previous measure was biased by the change in the size of the UI extension policy in a recession relative to normal times.
		\item Our new measure more naturally removes the bias to policies that redistribute from low to high-marginal utility households in normal times. Previously, we took away the welfare benefit in normal times of each policy. In our new measure, ANY marginal redistributive policy has no welfare benefit in normal times.
	\end{itemize}
	For more details on how we treat welfare, see section~4.3.
	
	\item \textbf{Robustness in a General Equilibrium Model}  The main results of this paper are presented in a partial equilibrium setup with aggregate demand effects that do not arise from general equilibrium effects. We think there are many advantages to studying the welfare and multiplier effects in this setting without embedding the model in general equilibrium.
	
	However, we now complement our analysis with a general equilibrium HANK and SAM model similar to \citet{Ravn2017}. This model is as standard as possible, but able to capture supply-side effects that are absent from the partial equilibrium setup. In this model we also introduce a fiscal rule to balance the government budget. We find that the consumption multipliers across horizons follow the same qualitative pattern as we have in our partial equilibrium analysis.
	
	The results from this HANK and SAM model are presented in section~5, and the details of the model are in Appendix~B.
	
      \item \textbf{Consumption drop upon expiry of unemployment benefits.} One referee suggested that we report the size of the drop in consumption for households who remain unemployed for long enough for unemployment benefits to expire; \citeauthor{ganongConsumer2019}'s found that the drop was much steeper than could be explained by a consumption model with standard features including a time preference rate calibrated to match, as best it can, the entire path of consumption pre- and post-expiry.  If we were to add the splurge to that baseline model, it would indeed make the drop in spending at UI expiration substantially steeper. However, our model is one in which agents have heterogeneous time preference factors.  In a little-noticed exercise later in the \citeauthor{ganongConsumer2019} paper, the authors examine a model with heterogeneous discount factors, and find that when the distribution of time preference rates is calibrated to match the consumption paths, the model goes a long way to explaining the sharp consumption drop.  The difference reflects the fact that in a model with a single homogeneous time preference rate, the model must match a pattern of consumption that implies that the great majority of consumers will have a substantial buffer stock of savings at the time when UI benefits expire.  But a model with heterogeneous time preference rates has an additional degree of freedom to have heterogeneous MPCs at the point of expiry.  In the HA-prefs model, the more impatient consumers will have a larger drop in consumption upon becoming unemployed, but the more patient ones will have a smaller initial drop. The total pattern of consumption decline however can match the observed total decline.  The difference is that by the end of the UI-covered period, some households have run down their wealth more drastically than others, and those impatient households will have a much bigger drop in consumption than in the identical-prefs model.  In other words, there is much less of a puzzle in their robustness test than in their headline model - and our model is already like that robustness variant.
        We now plot the path of income and spending around the expiry of unemployment benefits in our model in section~3.3.3 (and for the version of the model without splurge consumption in Appendix~A.2), and directly compare this to \citeauthor{ganongConsumer2019}'s results.
	% [x] CDC has elaborated
	
	
\end{itemize}
