\begin{itemize}
	\item \textbf{The Splurge} We have added section \colorbox{yellow}{XX} and the Appendix \colorbox{yellow}{XX} to discuss the implications of the splurge for our model. 
	
	To do so we have reestimated the US model in a setting where the splurge is imposed to be zero. We show that with a wider distribution of discount factors, the model is able to account for the empirically observed liquid wealth distribution in the US, while also matching fairly well untargeted moments on the consumption dynamics in response to transitory income shocks and during unemployment spells, albeit not as well as the model allowing for a splurge. 
	
	We then simulate the three fiscal policies in the reestimated models, showing that our multiplier and welfare metrics for the policies differ only marginally between the model with and without the splurge. Importantly, the ranking of the policies also remains unaffected by the exclusion of splurge. While the splurge is thus helpful in matching available empirical evidence, it does not affect the main conclusions of our paper.
	
	For this reason we have also substantially rewritten our motivation in the introduction of the paper for why we include the splurge, pointing out that the splurge acts as a stand-in for competing theories for why agents may spend more out of transitory income shocks than suggested by a simple model in which agents solely maximize utility. As you point out, our finding that the splurge does not matter too greatly for consumption dynamics to affect our policy conclusions, is an interesting finding in itself. As the splurge still provides a slightly better fit of the empirical targets, we have opted to keep the version with the splurge as our baseline model.
	
	
	\item \textbf{Welfare Measure} We have completely overhauled our section on welfare and have introduced a new measure that we think best captures the idea that we want to measure welfare gains from carrying out each policy during a recession, but give no benefit in normal times to policies that in our model would increase welfare through redistribution. 
	
	Our welfare measure weights the felicity of a household at time $t$ by the marginal utility of the same household in a counterfactual simulation in which neither the recession occurred nor the fiscal policy was implemented. This weighting scheme means that in normal times the marginal benefit to a social planner of moving a dollar of consumption from one household at one time period to another household at the same or different time period is zero. Hence, in normal times, any re-distributive policy has zero marginal benefit. However, in a recession when the average marginal utility is higher than in normal times, there can be welfare benefits to government borrowing to allow households to consume more during the recession.
	
	Our new welfare measure leads to the same qualitative conclusions as in the previous version of the paper. It improves on the previous measure in several ways:
	\begin{itemize}
		\item The new measure does not scale with the size of the fiscal stimulus. We divide by the net present value of the fiscal policy so the measure is a `bang-bang-for-the-buck' measure. As pointed out by referee 2, our previous measure was biased by the change in the size of the UI extension policy in a recession relative to normal times.
		\item Our new measure more naturally removes the bias to policies that redistribute from high to low-marginal utility households in normal times. Previously, we took away the welfare benefit in normal times of each policy. In our new measure, ANY marginal redistributive policy has no welfare benefit in normal times.
	\end{itemize}
	For more details on how we treat welfare, see section \colorbox{yellow}{XX}.
	
	\item \textbf{Robustness in a General Equilibrium Model}  The main results of this paper are presented in a partial equilibrium setup with aggregate demand effects that do not arise from a general equilibrium setup. We think there are many advantages to studying the welfare and multiplier effects in this setting without embedding the model in general equilibrium.
	
	However,  we now complement our analysis with a  general equilibrium HANK model, as standard as possible, but able to capture supply-side effects that are absent from the partial equilibrium setup and to introduce a fiscal rule to balance the government budget. We find that the consumption multipliers across horizons follow the same qualitative pattern as we have in our partial equilibrium analysis.
	
	The write up of this HANK model is found in section \colorbox{yellow}{XX}
	
	
\end{itemize}