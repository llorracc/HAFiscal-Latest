% -*- mode: LaTeX; TeX-PDF-mode: t; -*-  # Config emacs auctex

% allow latex to find custom stuff
% Add the listed directories to the search path
% (allows easy moving of files around later)
% these paths are searched AFTER local config kpsewhich

% *.sty, *.cls
\makeatletter
\def\input@path{{@resources/texlive/texmf-local/tex/latex//}
        ,{@resources/texlive/latex//}
        ,{@local//}
        }
\makeatother
\makeatletter
\def\bibinput@path{{@resources/texlive/texmf-local/tex/latex//}
        ,{@resources/texlive/latex//},
        ,{@local//}
        }
\makeatother
  

\documentclass[12pt,letterpaper,english]{article}
\usepackage[round, authoryear]{natbib}
\usepackage{floatrow}
\renewcommand{\floatpagefraction}{.99}
\newfloatcommand{capbtabbox}{table}[][\FBwidth]
\usepackage{eurosym}
\usepackage{graphicx}
\usepackage{setspace}
\usepackage{geometry}
\usepackage{bbm}
\usepackage{hyperref}
\usepackage[titletoc]{appendix}
\usepackage{graphicx}
\newcommand{\argmin}{\arg\!\min}
\usepackage{amsmath, amssymb,amsthm,mathtools,dsfont}
\usepackage{algorithmic}
\usepackage{booktabs}
\usepackage{setspace}
\usepackage{enumerate}
\usepackage{enumitem}
\usepackage[flushleft]{threeparttable}
\usepackage{rotating}
\usepackage{float}
\usepackage{tabulary}
\usepackage{ragged2e}
\usepackage{rotating}
\usepackage{epstopdf}
\usepackage[labelfont=bf]{caption}
\usepackage{array}
\newcolumntype{C}[1]{>{\centering\let\newline\\\arraybackslash\hspace{0pt}}m{#1}}
\usepackage{subfig}
\usepackage{placeins}
\usepackage{pxfonts}
\usepackage{xcolor}
\usepackage{svg}
\usepackage{multirow}
\usepackage{verbatim}
\usepackage{soul}
\sethlcolor{yellow}
\hypersetup{
	colorlinks,
	linkcolor={red!50!black},
	citecolor={red!50!black},
	urlcolor={blue!80!black}
}
\textheight=23cm \textwidth=16.5cm \oddsidemargin=0cm
\evensidemargin=0cm \topmargin=-1.75cm
\setcounter{MaxMatrixCols}{10}
\makeatletter
\g@addto@macro\@floatboxreset\centering
\makeatother
\setlength\floatsep{2\baselineskip plus 3pt minus 2pt}
\setlength\textfloatsep{2\baselineskip plus 3pt minus 2pt}
\setlength\intextsep{2\baselineskip plus 3pt minus 2 pt}
\newcommand*{\MyIndent}{\hspace*{0.5cm}}%
\usepackage[normalem]{ulem}
\newtheorem{theorem}{Proposition}
\newtheorem{definition}{Definition}
\newtheorem{proposition}{Proposition}
\newtheorem{corollary}{Corollary}
\newtheorem{example}{Example}
\newtheorem{lemma}{Lemma}
\newtheorem{remark}{Remark}
\newtheorem{assumption}{Assumptions}
\newtheorem{hypothesis}{Hypothesis}


% Uncomment the next line to use the harvard package with bibtex
%\usepackage[abbr]{harvard}

% This command determines the leading (vertical space between lines) in draft mode
% with 1.5 corresponding to "double" spacing.
%\draftSpacing{1.5}
%\newlength\TableWidth
\usepackage{array}
\newcolumntype{H}{>{\setbox0=\hbox\bgroup}c<{\egroup}@{}}
\newcommand{\Figures}{Figures/}
\newcommand{\Tables}{Tables/}
\usepackage{pifont}
\newcommand{\cmark}{\ding{51}}%
\newcommand{\xmark}{\ding{55}}%

\title{\textbf{Response to Editor\\ Quantitative Economics MS 2442 \\``Welfare and Spending Effects of \\ Consumption Stimulus Policies''}}
\author{Christopher D. Carroll, Edmund Crawley, William Du, \\ Ivan Frankovic, and H\aa kon Tretvoll}
\date{}

\begin{document}
	\onehalfspacing
	\maketitle
	
	\noindent Thank you for giving us the chance to resubmit our paper ``Welfare and Spending Effects of Consumption Stimulus Policies'' to Quantitative Economics. And thank you for your thoughtful comments and suggestions for how to improve our paper. They were all very useful to us in revising it. We believe the paper has improved greatly through the revision and we hope you agree. 
	
	In the following, we go through how we have dealt with the specific requests from you. For each request, we first repeat your comment in italics and then respond how we have dealt with them.
	
\begin{enumerate}
	
\item \textit{\textbf{Splurge.} Reading the current draft of the paper, one feels almost as if there is	no reason for including the ``splurge'' consumption component in the analysis. But I think this is unfair and maybe you should highlight this a bit more. For example, the model without splurge requires unreasonably low, in my opinion, discount factors	for some agents while the distribution looks a lot more reasonable when you allow for splurge.}

  \noindent \textbf{Response}

  A virtue of QE, compared to many other journals, is the importance QE attaches to getting \textit{quantitative} things right.  The original Krusell-Smith (1998) paper, for example, concluded that heterogeneity didn't make much difference to macroeconomics -- but that's because they got things \textit{quantitatively} wrong by calibrating their model without reference to the microeconomic facts (in particular, the distribution of wealth).

  We implicitly highlight this point in our emphasis on the importance of using a \href{https://llorracc.github.io/HAFiscal/#microeconomically-credible}{\textbf{microeconomically credible}} model.

  Since our last submission this fall, four new papers (incorporated in our revised lit review) have appeared (or come to our attention) measuring and theorizing about the phenomenon of what we now dub the `excess initial MPC.'  Several of these papers propose or speculate that incorporating the excess initial MPC might substantially change macro dynamics.

  In our particular context, if we had a model that did not match the `excess initial MPC,' any reader familiar with this hot topic in the consumption literature could reasonably wonder whether our results might be \textit{quantitatively} off (maybe substantially so) just as Krusell-Smith's were.

  We show that the answer to the question turns out to be that while the splurge that we introduced to match the excess initial MPC makes some difference, incorporating it does not turn out to fundamentally change the results.

  But that's an interesting point in itself: Maybe it means that, however robust the phenomenon, it may not be of first-order importance for macro dynamics.
  
  \textit{I also wonder how the splurge consumption approach compares to the ``infrequent'' consumption component introduced in \citet{melcangiStock}? On the surface, the two approaches look very different, but both can generate very	high MPCs. A very short discussion would suffice.}

  We have now added a reference to the \citet{melcangiStock} paper; it is one of the four papers mentioned above.  They provide an interesting and plausible story in which even wealthy people have occasional large spikes in spending, caused (for example) by tuition expenses for children, or necessary health-related expenditures. But in their paper, these `spending opportunities' occur independently of the income process.  For a person with little wealth, if there is a double coincidence -- a positive shock to income happens to occur at the same time that a spending opportunity happens, a person who otherwise could not take advantage of the opportunity might be able to afford it.

  But their model says that for wealthy people, the opportunity will be seized even if there is no positive income shock at the same time.  This is inconsistent with the robust finding that \href{llorracc.github.io/HAFiscal/#wealthy-high-MPC}{even wealthy households exhibit a high `initial MPC' out of transitory shocks}.

  (In any case, the word `splurge' does not apply naturally to expenses like tuition or emergency health care or many of the other plausible examples that come to mind; we would describe these as constituting a `spike' in spending but not a splurge.)
  
  Our addition to footnote 9 reads: ``In contrast, the “infrequent consumption good” by \citet{melcangiStock} aims at accounting for high saving rates among high-income households during normal times and high consumption during episodes where the infrequent consumption good becomes available (such as high-end health care, education expenses or bequests). The high consumption is thus not triggered by a transitory income shock but by rare consumption opportunities.''

\item \textit{\textbf{Computation.} Referee 1 (the report that starts ``In this revision,'') asks why you do not use the non-linear transition method of \citet{boppart2018exploiting} to accommodate a large shock? I think a short sentence in the paper on this is ok.}

\noindent \textbf{Response.} We have added a footnote in the HANK section: ``One approach to overcome this limitation, which could be used in future work, is described in \cite{boppart2018exploiting}.''

\item \textit{\textbf{The welfare measure.} Referee 2 (the report that starts ``Overall, I view'' would like you to discuss your welfare measure and motivate better why you do not use a ``simple'' utilitarian measure. I agree with the referee. I was initially thinking that	your measure would be similar to using Negishi weights, but I do not think this is true (and it would also be odd to use Negishi weights under incomplete markets). I am not sure either whether your measure allows one to figure whether the policies are PPI (Potentially Pareto Improving) when allowing for redistribution, but I don't think so either. I do not want you to change the welfare measure, but just to include a paragraph discussing the reasons for your choice.}
	
\noindent \textbf{Response.} 

Thank you for this suggestion and for drawing our attention to Negishi weights which we were not familiar with. As per your suggestion with have added a paragraph (with 2 footnotes) to the welfare section:

``As with all social welfare measures, ours is not without ethical issues.  We have chosen our welfare measure over one with equal weights because an equal-weights measure would be increasing with the size of any redistributive policy.$^1$ However, similar to Negishi weights, our welfare measure gives greater weight to households that are well off.$^2$  Furthermore, our welfare measure distinguishes between households that would have suffered unemployment in normal times and households that are made unemployed as a result of the recession—--giving the later a higher weight in the social welfare function.''

Footnote 1: Using a version of an equal-weights measure results in an even greater welfare benefit to extended unemployment insurance---see the previously distributed draft of this paper, \cite{carroll2023welfare}.  However, because the size of the extended unemployment benefits policy is much larger in a recession compared to normal times, while the size of the other two policies does not change significantly in a recession, this equal-weights measure almost mechanically favored the extended unemployment benefits policy.

Footnote 2: Negishi weights have been used in the climate literature as a way to separate the welfare benefits of climate mitigation policies from broader questions about global income redistribution. Our problem of separating the welfare benefits of recession mitigation policies from income redistribution in normal times is similar, but complicated by our incomplete markets setup. With complete markets, under which there is no potential benefit to redistributing consumption across time for any individual household, our measure is identical to Negishi weights.

\item \textit{\textbf{The HANK\&SAM model} All three referees and myself appreciate the introduction of the HANK\&SAM model, but some issues also come up:}
	\begin{itemize}
		\item \textit{I think most readers would struggle to get much out of the model description in Section~5. Please extend it a bit.}
		
		\noindent \textbf{Response.} 
		
		\item \textit{When comparing Figures 4 and 8, the size of the effects in the PE and the GE models seem very different, but I think this is due to a scaling issue. Please check.}
		
		\noindent \textbf{Response.} Yes - this was a mistake at our end. We have now scaled the stimulus check appropriately. The UI extension is smaller in our HANK model because fewer people are unemployed in steady state---the initial condition for our HANK MIT shock---than in the recession---the initial condition for our main analysis. We have extended footnote 36: ``Note that the dynamics of the UI extension IRF are somewhat faster acting.
		This is because, under the recession that we study in the partial equilibrium analysis, the large mass of newly-unemployed households do not start receiving extended UI for six months. Furthermore, the magnitude of the UI policy is lower because fewer people are unemployed in steady state---the initial condition for our HANK shocks.''
		
		\item \textit{Referee 1 suggests that the HANK\&SAM model should have been the baseline model and, in comment Additional (a), points out that you seem to dismiss the model for no reason. I do not want you to change the baseline model, but	the sentence in the introduction highlighted by the referee seems odd having read the paper.}
		
		\noindent \textbf{Response.} We agree and we have removed the offending sentence.
		
		\item \textit{I found the citation/credit for the HANK\&SAM model a bit odd since Vincent Sterk and I were pushing this line of work for many years in our 2017 JME	paper as well as in our JEEA 2021 paper. But you can keep the citations as they are, I am probably just over-sensitive to the issue because it took us 6 years to publish the 2017 paper.}
		
		\noindent \textbf{Response.} 
	\end{itemize}
	
\item \textit{\textbf{Fiscal policy evaluation.} There is by now a lot of work on fiscal multipliers, their determinants, interpretation etc. It would thus be unreasonable to expect you to fully relate your paper to extant work. However, one paper that I have seen presented which shares a lot of similarities with your's is \citet{broer2024stimulus}. These authors seem to find somewhat different results from you. It would be good	to have a very short discussion of this.}	

\noindent \textbf{Response.} 

\item \textit{\textbf{Tables and Figures.} Please make sure all tables and figures are self-explanatory.} 

\noindent \textbf{Response.} 

\end{enumerate}

\bigskip

\noindent Thank you again for your careful advice on our paper. We hope you find our revision satisfactory.

\bibliographystyle{econark}
\bibliography{../../../../HAFiscal.bib,response}

\end{document}
