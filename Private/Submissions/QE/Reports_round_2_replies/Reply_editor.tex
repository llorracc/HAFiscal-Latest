% -*- mode: LaTeX; TeX-PDF-mode: t; -*-  # Config emacs auctex

% allow latex to find custom stuff
% Add the listed directories to the search path
% (allows easy moving of files around later)
% these paths are searched AFTER local config kpsewhich

% *.sty, *.cls
\makeatletter
\def\input@path{{@resources/texlive/texmf-local/tex/latex//}
        ,{@resources/texlive/latex//}
        ,{@local//}
        }
\makeatother
\makeatletter
\def\bibinput@path{{@resources/texlive/texmf-local/tex/latex//}
        ,{@resources/texlive/latex//},
        ,{@local//}
        }
\makeatother
  

\documentclass[12pt,letterpaper,english]{article}
\usepackage[round, authoryear]{natbib}
\usepackage{floatrow}
\renewcommand{\floatpagefraction}{.99}
\newfloatcommand{capbtabbox}{table}[][\FBwidth]
\usepackage{eurosym}
\usepackage{graphicx}
\usepackage{setspace}
\usepackage{geometry}
\usepackage{bbm}
\usepackage{hyperref}
\usepackage[titletoc]{appendix}
\usepackage{graphicx}
\newcommand{\argmin}{\arg\!\min}
\usepackage{amsmath, amssymb,amsthm,mathtools,dsfont}
\usepackage{algorithmic}
\usepackage{booktabs}
\usepackage{setspace}
\usepackage{enumerate}
\usepackage{enumitem}
\usepackage[flushleft]{threeparttable}
\usepackage{rotating}
\usepackage{float}
\usepackage{tabulary}
\usepackage{ragged2e}
\usepackage{rotating}
\usepackage{epstopdf}
\usepackage[labelfont=bf]{caption}
\usepackage{array}
\newcolumntype{C}[1]{>{\centering\let\newline\\\arraybackslash\hspace{0pt}}m{#1}}
\usepackage{subfig}
\usepackage{placeins}
\usepackage{pxfonts}
\usepackage{xcolor}
\usepackage{svg}
\usepackage{multirow}
\usepackage{verbatim}
\usepackage{soul}
\sethlcolor{yellow}
\hypersetup{
	colorlinks,
	linkcolor={red!50!black},
	citecolor={red!50!black},
	urlcolor={blue!80!black}
}
\textheight=23cm \textwidth=16.5cm \oddsidemargin=0cm
\evensidemargin=0cm \topmargin=-1.75cm
\setcounter{MaxMatrixCols}{10}
\makeatletter
\g@addto@macro\@floatboxreset\centering
\makeatother
\setlength\floatsep{2\baselineskip plus 3pt minus 2pt}
\setlength\textfloatsep{2\baselineskip plus 3pt minus 2pt}
\setlength\intextsep{2\baselineskip plus 3pt minus 2 pt}
\newcommand*{\MyIndent}{\hspace*{0.5cm}}%
\usepackage[normalem]{ulem}
\newtheorem{theorem}{Proposition}
\newtheorem{definition}{Definition}
\newtheorem{proposition}{Proposition}
\newtheorem{corollary}{Corollary}
\newtheorem{example}{Example}
\newtheorem{lemma}{Lemma}
\newtheorem{remark}{Remark}
\newtheorem{assumption}{Assumptions}
\newtheorem{hypothesis}{Hypothesis}


% Uncomment the next line to use the harvard package with bibtex
%\usepackage[abbr]{harvard}

% This command determines the leading (vertical space between lines) in draft mode
% with 1.5 corresponding to "double" spacing.
%\draftSpacing{1.5}
%\newlength\TableWidth
\usepackage{array}
\newcolumntype{H}{>{\setbox0=\hbox\bgroup}c<{\egroup}@{}}
\newcommand{\Figures}{Figures/}
\newcommand{\Tables}{Tables/}
\usepackage{pifont}
\newcommand{\cmark}{\ding{51}}%
\newcommand{\xmark}{\ding{55}}%

\title{\textbf{Response to Editor\\ Quantitative Economics MS 2442 \\``Welfare and Spending Effects of \\ Consumption Stimulus Policies''}}
\author{Christopher D. Carroll, Edmund Crawley, William Du, \\ Ivan Frankovic, and H\aa kon Tretvoll}
\date{}

\begin{document}
	\onehalfspacing
	\maketitle
	
	\noindent Thank you for giving us the chance to resubmit our paper ``Welfare and Spending Effects of Consumption Stimulus Policies'' to Quantitative Economics. And thank you for your thoughtful comments and suggestions for how to improve our paper. They were all very useful to us in revising it. We believe the paper has improved greatly through the revision and we hope you agree. 
	
	In the following, we first summarize the main changes we have made based on your and the referees' suggestions. Thereafter, we go through how we have dealt with the specific requests from you. For each request, we first repeat your comment in italics and then respond how we have dealt with them.
	
	\section{Summary of Main Changes}
	
\begin{itemize}
	\item \textbf{The Splurge}
	\item \textbf{Welfare Measure} We have completely overhauled our section on welfare and have introduced a new measure that we think best captures the idea that we want to measure welfare gains from carrying out each policy during a recession, but give no benefit in normal times to policies that in our model would increase welfare through redistribution. 
	
	Our welfare measure weights the felicity of a household at time $t$ by the marginal utility of the same household in a counterfactual simulation in which neither the recession occurred nor the fiscal policy was implemented. This weighting scheme means that in normal times the marginal benefit to a social planner of moving a dollar of consumption from one household at one time period to another household at the same or different time period is zero. Hence, in normal times, any re-distributive policy has zero marginal benefit. However, in a recession when the average marginal utility is higher than in normal times, there can be welfare benefits to government borrowing to allow households to consume more during the recession.
	
	Our new welfare measure leads to the same qualitative conclusions as in the previous version of the paper. It improves on the previous measure in several ways:
	\begin{itemize}
		\item The new measure does not scale with the size of the fiscal stimulus. We divide by the net present value of the fiscal policy so the measure is a `bang-bang-for-the-buck' measure. As pointed out by referee 2, our previous measure was biased by the change in the size of the UI extension policy in a recession relative to normal times.
		\item Our new measure more naturally removes the bias to policies that redistribute from high to low-marginal utility households in normal times. Previously, we took away the welfare benefit in normal times of each policy. In our new measure, ANY marginal redistributive policy has no welfare benefit in normal times.
		\item One benefit of removing the splurge from our household behavior is that our welfare measure now matches with the utility function in the households' optimization problem.
	\end{itemize}
	For more details on how we treat welfare, see section \colorbox{yellow}{XX}.
	
	\item \textbf{Robustness in a General Equilibrium Model}  The main results of this paper are presented in a partial equilibrium setup with aggregate demand effects that do not arise from a general equilibrium setup. We think there are many advantages to studying the welfare and multiplier effects in this setting without embedding the model in general equilibrium.
	
	However,  we now complement our analysis with a  general equilibrium HANK model, as standard as possible, but able to capture supply-side effects that are absent from the partial equilibrium setup and to introduce a fiscal rule to balance the government budget. We find that the consumption multipliers across horizons follow the same qualitative pattern as we have in our partial equilibrium analysis.
	
	The write up of this HANK model is found in section \colorbox{yellow}{XX}
	
	
\end{itemize}
	
	\newpage 
	
\section{Remaining issues}
\begin{enumerate}
	
\item \textit{\textbf{Splurge.} Reading the current draft of the paper, one feels almost as if there is	no reason for including the ``splurge'' consumption component in the analysis. But I think this is unfair and maybe you should highlight this a bit more. For example, the model without splurge requires unreasonably low, in my opinion, discount factors	for some agents while the distribution looks a lot more reasonable when you allow for splurge.}

\textit{I also wonder how the splurge consumption approach compares to the ``infrequent'' consumption component introduced in \citet{melcangi2024stock}? On the surface, the two approaches look very different, but both can generate very	high MPCs. A very short discussion would suffice.}

\noindent \textbf{Response.} 

\item \textit{\textbf{Computation.} Referee 1 (the report that starts ``In this revision,'') asks why you do not use the non-linear transition method of \citet{boppart2018exploiting} to accommodate a large shock? I think a short sentence in the paper on this is ok.}

\noindent \textbf{Response.} 

\item \textit{\textbf{The welfare measure.} Referee 2 (the report that starts ``Overall, I view'' would like you to discuss your welfare measure and motivate better why you do not use a ``simple'' utilitarian measure. I agree with the referee. I was initially thinking that	your measure would be similar to using Negishi weights, but I do not think this is true (and it would also be odd to use Negishi weights under incomplete markets). I am not sure either whether your measure allows one to figure whether the policies are PPI (Potentially Pareto Improving) when allowing for redistribution, but I don't think so either. I do not want you to change the welfare measure, but just to include a paragraph discussing the reasons for your choice.}
	
\noindent \textbf{Response.} 

\item \textit{\textbf{The HANK\&SAM model} All three referees and myself appreciate the introduction of the HANK\&SAM model, but some issues also come up:}
	\begin{itemize}
		\item \textit{I think most readers would struggle to get much out of the model description in Section~5. Please extend it a bit.}
		
		\noindent \textbf{Response.} 
		
		\item \textit{When comparing Figures 4 and 8, the size of the effects in the PE and the GE models seem very different, but I think this is due to a scaling issue. Please check.}
		
		\noindent \textbf{Response.} 
		
		\item \textit{Referee 1 suggests that the HANK\&SAM model should have been the baseline model and, in comment Additional (a), points out that you seem to dismiss the model for no reason. I do not want you to change the baseline model, but	the sentence in the introduction highlighted by the referee seems odd having read the paper.}
		
		\noindent \textbf{Response.} 
		
		\item \textit{I found the citation/credit for the HANK\&SAM model a bit odd since Vincent Sterk and I were pushing this line of work for many years in our 2017 JME	paper as well as in our JEEA 2021 paper. But you can keep the citations as they are, I am probably just over-sensitive to the issue because it took us 6 years to publish the 2017 paper.}
		
		\noindent \textbf{Response.} 
	\end{itemize}
	
\item \textit{\textbf{Fiscal policy evaluation.} There is by now a lot of work on fiscal multipliers, their determinants, interpretation etc. It would thus be unreasonable to expect you to fully relate your paper to extant work. However, one paper that I have seen presented which shares a lot of similarities with your's is \citet{broer2024stimulus}. These authors seem to find somewhat different results from you. It would be good	to have a very short discussion of this.}	

\noindent \textbf{Response.} 

\item \textit{\textbf{Tables and Figures.} Please make sure all tables and figures are self-explanatory.} 

\noindent \textbf{Response.} 

\end{enumerate}

\bigskip

\noindent Thank you again for your careful advice on our paper. We hope you find our revision satisfactory.

\bibliographystyle{econark}
\bibliography{../../../../HAFiscal.bib,response}

\end{document}
