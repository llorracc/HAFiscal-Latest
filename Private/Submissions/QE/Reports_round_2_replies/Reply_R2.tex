% -*- mode: LaTeX; TeX-PDF-mode: t; -*-  # Config emacs auctex

% allow latex to find custom stuff
% Add the listed directories to the search path
% (allows easy moving of files around later)
% these paths are searched AFTER local config kpsewhich

% *.sty, *.cls
\makeatletter
\def\input@path{{@resources/texlive/texmf-local/tex/latex//}
        ,{@resources/texlive/latex//}
        ,{@local//}
        }
\makeatother
\makeatletter
\def\bibinput@path{{@resources/texlive/texmf-local/tex/latex//}
        ,{@resources/texlive/latex//},
        ,{@local//}
        }
\makeatother
  

\documentclass[12pt,letterpaper,english]{article}
\usepackage[round, authoryear]{natbib}
\usepackage{floatrow}
\renewcommand{\floatpagefraction}{.99}
\newfloatcommand{capbtabbox}{table}[][\FBwidth]
\usepackage{eurosym}
\usepackage{graphicx}
\usepackage{setspace}
\usepackage{geometry}
\usepackage{bbm}
\usepackage{hyperref}
\usepackage[titletoc]{appendix}
\usepackage{graphicx}
\newcommand{\argmin}{\arg\!\min}
\usepackage{amsmath, amssymb,amsthm,mathtools,dsfont}
\usepackage{algorithmic}
\usepackage{booktabs}
\usepackage{setspace}
\usepackage{enumerate}
\usepackage{enumitem}
\usepackage[flushleft]{threeparttable}
\usepackage{rotating}
\usepackage{float}
\usepackage{tabulary}
\usepackage{ragged2e}
\usepackage{rotating}
\usepackage{epstopdf}
\usepackage[labelfont=bf]{caption}
\usepackage{array}
\newcolumntype{C}[1]{>{\centering\let\newline\\\arraybackslash\hspace{0pt}}m{#1}}
\usepackage{subfig}
\usepackage{placeins}
\usepackage{pxfonts}
\usepackage{xcolor}
\usepackage{svg}
\usepackage{multirow}
\usepackage{verbatim}
\usepackage{soul}
\sethlcolor{yellow}
\hypersetup{
	colorlinks,
	linkcolor={red!50!black},
	citecolor={red!50!black},
	urlcolor={blue!80!black}
}
\textheight=23cm \textwidth=16.5cm \oddsidemargin=0cm
\evensidemargin=0cm \topmargin=-1.75cm
\setcounter{MaxMatrixCols}{10}
\makeatletter
\g@addto@macro\@floatboxreset\centering
\makeatother
\setlength\floatsep{2\baselineskip plus 3pt minus 2pt}
\setlength\textfloatsep{2\baselineskip plus 3pt minus 2pt}
\setlength\intextsep{2\baselineskip plus 3pt minus 2 pt}
\newcommand*{\MyIndent}{\hspace*{0.5cm}}%
\usepackage[normalem]{ulem}
\newtheorem{theorem}{Proposition}
\newtheorem{definition}{Definition}
\newtheorem{proposition}{Proposition}
\newtheorem{corollary}{Corollary}
\newtheorem{example}{Example}
\newtheorem{lemma}{Lemma}
\newtheorem{remark}{Remark}
\newtheorem{assumption}{Assumptions}
\newtheorem{hypothesis}{Hypothesis}


% Uncomment the next line to use the harvard package with bibtex
%\usepackage[abbr]{harvard}

% This command determines the leading (vertical space between lines) in draft mode
% with 1.5 corresponding to "double" spacing.
%\draftSpacing{1.5}
%\newlength\TableWidth
\usepackage{array}
\newcolumntype{H}{>{\setbox0=\hbox\bgroup}c<{\egroup}@{}}
\newcommand{\Figures}{Figures/}
\newcommand{\Tables}{Tables/}
\usepackage{pifont}
\newcommand{\cmark}{\ding{51}}%
\newcommand{\xmark}{\ding{55}}%

\title{\textbf{Response to Referee 2\\ Quantitative Economics MS 2442 \\``Welfare and Spending Effects of \\ Consumption Stimulus Policies''}}
\author{Christopher D. Carroll, Edmund Crawley, William Du, \\ Ivan Frankovic, and H\aa kon Tretvoll}
\date{}

\begin{document}
\onehalfspacing
\maketitle
	
\noindent Thank you for your thoughtful comments and suggestions on our paper ``Welfare and Spending Effects of Consumption Stimulus Policies''. They were all very useful to us in revising the paper. We hope you agree that the paper has improved. In the following, we state each of your comments in italics and provide point-by-point responses to them.
	
\section{Comments}
\begin{itemize}

	\item \textit{\textbf{Welfare measure.} The new welfare measure is quite elegant and satisfies nice properties, but it also makes some strong ethical choices.}
		
	\textit{Here is one way to explain the welfare measure in words: we take a revealed preference approach to policy, keeping individual-specific welfare weights fixed.}
		
	\textit{Since the policy maker is letting an unemployed individual $i$ suffer in an expansion, the policy maker must attach little weight to $i$. However, when an individual $j$ becomes unemployed in a recession who would have been	employed otherwise, then there is scope for policy.}
	
	\textit{The welfare measure attaches a larger weight to individuals who become unemployed because of the recession, and it becomes more attractive to extend unemployment insurance. On layman ethical grounds I find this welfare measure somewhat repulsive. We put a larger welfare measure on some unemployed than other unemployed because of the fact that they became unemployed due to the recession.}
	
	\textit{Contrast this with the previous version’s welfare measure (at least in the ‘bang for the buck’ version suggested by me). With such a welfare measure, the value of extending unemployment insurance is the utilitarian amount of misery it alleviates per dollar spent, benchmarked against the misery it alleviates in an expansion. Personally, I find such a formulation of welfare more appealing.}

	\begin{enumerate}
		\item \textit{I am not necessarily suggesting changing the welfare measure (although I would!) but I do think the authors should highlight that these are complicated ethical issues, perhaps by explaining that the welfare measure implicitly makes a distinction between unemployed because of the recession (‘deserving poor’) and unemployed regardless (‘undeserving poor’).}
	\end{enumerate}
	
	\noindent \textbf{Response.} Thank you for highlighting some of the ethical concerns with our welfare measure. We have added a paragraph highlighting these concerns and explaining our motivations:
	
	``As with all social welfare measures, ours is not without ethical issues.  We have chosen our welfare measure over one with equal weights because an equal-weights measure would be increasing with the size of any redistributive policy.$^1$ However, similar to Negishi weights, our welfare measure gives greater weight to households that are well off.$^2$  Furthermore, our welfare measure distinguishes between households that would have suffered unemployment in normal times and households that are made unemployed as a result of the recession—--giving the latter a higher weight in the social welfare function.''
	
	Footnote 1: Using a version of an equal-weights measure results in an even greater welfare benefit to extended unemployment insurance---see the previously distributed draft of this paper, \cite{carroll2023welfare}.  However, because the size of the extended unemployment benefits policy is much larger in a recession compared to normal times, while the size of the other two policies does not change significantly in a recession, this equal-weights measure almost mechanically favored the extended unemployment benefits policy.
	
	Footnote 2: Negishi weights have been used in the climate literature as a way to separate the welfare benefits of climate mitigation policies from broader questions about global income redistribution. Our problem of separating the welfare benefits of recession mitigation policies from income redistribution in normal times is similar, but complicated by our incomplete markets setup. With complete markets, under which there is no potential benefit to redistributing consumption across time for any individual household, our measure is identical to Negishi weights.
 
	
	\item \textit{\textbf{General equilibrium HANK-SAM.} Overall, the exercise seems fine but
		could benefit from a bit more polishing, see below.}
		
	\begin{enumerate}[start=2]
		\item \textit{Why are the consumption deviations in Figure 8 of such different orders			of magnitude? For example, the stimulus effect has an impact effect on consumption of 2-3\% in Figure~4 but 0.10\% in Figure~8. Some necessary details for interpreting Figure~8 is missing, e.g., the sizes of the quantitative experiments.}

		\noindent \textbf{Response.} This was a mistake at our end. We have now scaled the stimulus check appropriately. The UI extension is smaller in our HANK model because fewer people are unemployed in steady state---the initial condition for our HANK MIT shock---than in the recession---the initial condition for our main analysis. We have extended footnote 36: ``Note that the dynamics of the UI extension IRF are somewhat faster acting.
		This is because, under the recession that we study in the partial equilibrium analysis, the large mass of newly-unemployed households do not start receiving extended UI for six months. Furthermore, the magnitude of the UI policy is lower because fewer people are unemployed in steady state---the initial condition for our HANK shocks.'' 
		
		\item \textit{The captions for Figure~7 and Figure~9 should be written so that it is clear that they are referring to the same experiment. Further, it would be nice if the line features in the plots were such that it is easy for a color-blind reader to tell the tax cut and the check apart.}
		
		\noindent \textbf{Response.}  	The captions should now be self explanatory. We have removed the original figure 7 and integrated into figure 4. Furthermore, we have changed the linestyles and colors of the figures to make them easier for a color-blind reader to read.
			
	\end{enumerate}
	

	
\end{itemize}

	
	\bigskip
	
	\noindent Thank you again for your careful advice on our paper. We hope you find our revision satisfactory.
	
\bibliographystyle{econark}
\bibliography{../../../../HAFiscal.bib,response}
\end{document}
