% -*- mode: LaTeX; TeX-PDF-mode: t; -*-  # Config emacs auctex

% allow latex to find custom stuff
% Add the listed directories to the search path
% (allows easy moving of files around later)
% these paths are searched AFTER local config kpsewhich

% *.sty, *.cls
\makeatletter
\def\input@path{{@resources/texlive/texmf-local/tex/latex//}
        ,{@resources/texlive/latex//}
        ,{@local//}
        }
\makeatother
\makeatletter
\def\bibinput@path{{@resources/texlive/texmf-local/tex/latex//}
        ,{@resources/texlive/latex//},
        ,{@local//}
        }
\makeatother
  

\documentclass[12pt,letterpaper,english]{article}
\usepackage[round, authoryear]{natbib}
\usepackage{floatrow}
\renewcommand{\floatpagefraction}{.99}
\newfloatcommand{capbtabbox}{table}[][\FBwidth]
\usepackage{eurosym}
\usepackage{graphicx}
\usepackage{setspace}
\usepackage{geometry}
\usepackage{bbm}
\usepackage{hyperref}
\usepackage[titletoc]{appendix}
\usepackage{graphicx}
\newcommand{\argmin}{\arg\!\min}
\usepackage{amsmath, amssymb,amsthm,mathtools,dsfont}
\usepackage{algorithmic}
\usepackage{booktabs}
\usepackage{setspace}
\usepackage{enumerate}
\usepackage{enumitem}
\usepackage[flushleft]{threeparttable}
\usepackage{rotating}
\usepackage{float}
\usepackage{tabulary}
\usepackage{ragged2e}
\usepackage{rotating}
\usepackage{epstopdf}
\usepackage[labelfont=bf]{caption}
\usepackage{array}
\newcolumntype{C}[1]{>{\centering\let\newline\\\arraybackslash\hspace{0pt}}m{#1}}
\usepackage{subfig}
\usepackage{placeins}
\usepackage{pxfonts}
\usepackage{xcolor}
\usepackage{svg}
\usepackage{multirow}
\usepackage{verbatim}
\usepackage{soul}
\sethlcolor{yellow}
\hypersetup{
	colorlinks,
	linkcolor={red!50!black},
	citecolor={red!50!black},
	urlcolor={blue!80!black}
}
\textheight=23cm \textwidth=16.5cm \oddsidemargin=0cm
\evensidemargin=0cm \topmargin=-1.75cm
\setcounter{MaxMatrixCols}{10}
\makeatletter
\g@addto@macro\@floatboxreset\centering
\makeatother
\setlength\floatsep{2\baselineskip plus 3pt minus 2pt}
\setlength\textfloatsep{2\baselineskip plus 3pt minus 2pt}
\setlength\intextsep{2\baselineskip plus 3pt minus 2 pt}
\newcommand*{\MyIndent}{\hspace*{0.5cm}}%
\usepackage[normalem]{ulem}
\newtheorem{theorem}{Proposition}
\newtheorem{definition}{Definition}
\newtheorem{proposition}{Proposition}
\newtheorem{corollary}{Corollary}
\newtheorem{example}{Example}
\newtheorem{lemma}{Lemma}
\newtheorem{remark}{Remark}
\newtheorem{assumption}{Assumptions}
\newtheorem{hypothesis}{Hypothesis}


% Uncomment the next line to use the harvard package with bibtex
%\usepackage[abbr]{harvard}

% This command determines the leading (vertical space between lines) in draft mode
% with 1.5 corresponding to "double" spacing.
%\draftSpacing{1.5}
%\newlength\TableWidth
\usepackage{array}
\newcolumntype{H}{>{\setbox0=\hbox\bgroup}c<{\egroup}@{}}
\newcommand{\Figures}{Figures/}
\newcommand{\Tables}{Tables/}
\usepackage{pifont}
\newcommand{\cmark}{\ding{51}}%
\newcommand{\xmark}{\ding{55}}%

\title{\textbf{Response to Referee 3 \\ Quantitative Economics MS 2442 \\``Welfare and Spending Effects of \\ Consumption Stimulus Policies''}}
\author{Christopher D. Carroll, Edmund Crawley, William Du, \\ Ivan Frankovic, and H\aa kon Tretvoll}
\date{}

\begin{document}
	\onehalfspacing
	\maketitle
	
	\noindent Thank you for your thoughtful comments and suggestions on our paper ``Welfare and Spending Effects of Consumption Stimulus Policies''. They were all very useful to us in revising the paper. We hope you agree that the paper has improved. In the following, we summarize the main changes we have made based on your, the other referees', and the editor's suggestions. Thereafter, we state each of your comments in italics and provide point-by-point responses to them.
	
	\section{Summary of Main Changes}
	
\begin{itemize}
	\item \textbf{The Splurge}
	\item \textbf{Welfare Measure} We have completely overhauled our section on welfare and have introduced a new measure that we think best captures the idea that we want to measure welfare gains from carrying out each policy during a recession, but give no benefit in normal times to policies that in our model would increase welfare through redistribution. 
	
	Our welfare measure weights the felicity of a household at time $t$ by the marginal utility of the same household in a counterfactual simulation in which neither the recession occurred nor the fiscal policy was implemented. This weighting scheme means that in normal times the marginal benefit to a social planner of moving a dollar of consumption from one household at one time period to another household at the same or different time period is zero. Hence, in normal times, any re-distributive policy has zero marginal benefit. However, in a recession when the average marginal utility is higher than in normal times, there can be welfare benefits to government borrowing to allow households to consume more during the recession.
	
	Our new welfare measure leads to the same qualitative conclusions as in the previous version of the paper. It improves on the previous measure in several ways:
	\begin{itemize}
		\item The new measure does not scale with the size of the fiscal stimulus. We divide by the net present value of the fiscal policy so the measure is a `bang-bang-for-the-buck' measure. As pointed out by referee 2, our previous measure was biased by the change in the size of the UI extension policy in a recession relative to normal times.
		\item Our new measure more naturally removes the bias to policies that redistribute from high to low-marginal utility households in normal times. Previously, we took away the welfare benefit in normal times of each policy. In our new measure, ANY marginal redistributive policy has no welfare benefit in normal times.
		\item One benefit of removing the splurge from our household behavior is that our welfare measure now matches with the utility function in the households' optimization problem.
	\end{itemize}
	For more details on how we treat welfare, see section \colorbox{yellow}{XX}.
	
	\item \textbf{Robustness in a General Equilibrium Model}  The main results of this paper are presented in a partial equilibrium setup with aggregate demand effects that do not arise from a general equilibrium setup. We think there are many advantages to studying the welfare and multiplier effects in this setting without embedding the model in general equilibrium.
	
	However,  we now complement our analysis with a  general equilibrium HANK model, as standard as possible, but able to capture supply-side effects that are absent from the partial equilibrium setup and to introduce a fiscal rule to balance the government budget. We find that the consumption multipliers across horizons follow the same qualitative pattern as we have in our partial equilibrium analysis.
	
	The write up of this HANK model is found in section \colorbox{yellow}{XX}
	
	
\end{itemize}
	
	\newpage 
	
\section{Comments}
\begin{itemize}
	
	\item \textit{I thank the authors for computing a model without the splurge factor. The results reported in the appendix suggest that such a model performs remarkably well in matching the most important features in the data, with the minor exception of the highest liquid wealth quartile. Therefore, I believe the paper could benefit from a more thorough motivation for including the splurge factor. For instance, is the distribution of $\beta$s in a model without the splurge factor unreasonable? It seems to me that the authors’ stated goal in the abstract (to ``assess the effectiveness of three fiscal stimulus policies'') could be achieved without the somewhat ad-hoc inclusion of the splurge parameter.}
	
	\noindent \textbf{Response.} 
	
	\item \textit{In my previous report, I raised a point regarding the calibration of labor market transition rates across different educational groups. The authors convincingly argue that heterogeneous $E$ to $U$ transitions fit the steady-state data best, as shown by \citet{elsby2010labor}. However, the same reference also indicates that while separation rates vary across educational groups, they are relatively unaffected by business cycles, whereas job-finding rates exhibit significant variation. From the manuscript, it is unclear to me whether the model reflects this characteristic. On page 9, it is stated that ``the employment transition matrix is adjusted so that unemployment remains at the new high level and the expected length of time for an unemployment spell increases,'' but it is unclear to what extent this adjustment	is driven by changes in $E$ to $U$ versus $U$ to $E$. Clarifying this would enhance the reader’s understanding.}	
	
	\noindent \textbf{Response.} 

	\item \textit{I very much appreciate the addition of the full HANK model as a robustness exercise. This	addition effectively demonstrates that the results are not merely an artifact of the partial equilibrium nature of the initial model, while still allowing readers to draw valuable insights from the latter.}

	\noindent \textbf{Response.} Thanks - we agree that the paper is stronger with the inclusion of this exercise. 
\end{itemize}

\bigskip

\noindent Thank you again for your careful advice on our paper. We hope you find our revision satisfactory.

\bibliographystyle{econark}
\bibliography{../../../../HAFiscal.bib,response}
\end{document}
