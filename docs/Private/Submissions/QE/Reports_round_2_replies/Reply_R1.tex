\documentclass[12pt,letterpaper,english]{article}
\usepackage[round, authoryear]{natbib}
\usepackage{floatrow}
\renewcommand{\floatpagefraction}{.99}
\newfloatcommand{capbtabbox}{table}[][\FBwidth]
\usepackage{eurosym}
\usepackage{graphicx}
\usepackage{setspace}
\usepackage{geometry}
\usepackage{bbm}
\usepackage{hyperref}
\usepackage[titletoc]{appendix}
\usepackage{graphicx}
\newcommand{\argmin}{\arg\!\min}
\usepackage{amsmath, amssymb,amsthm,mathtools,dsfont}
\usepackage{algorithmic}
\usepackage{booktabs}
\usepackage{setspace}
\usepackage{enumerate}
\usepackage{enumitem}
\usepackage[flushleft]{threeparttable}
\usepackage{rotating}
\usepackage{float}
\usepackage{tabulary}
\usepackage{ragged2e}
\usepackage{rotating}
\usepackage{epstopdf}
\usepackage[labelfont=bf]{caption}
\usepackage{array}
\newcolumntype{C}[1]{>{\centering\let\newline\\\arraybackslash\hspace{0pt}}m{#1}}
\usepackage{subfig}
\usepackage{placeins}
\usepackage{pxfonts}
\usepackage{xcolor}
\usepackage{svg}
\usepackage{multirow}
\usepackage{verbatim}
\usepackage{soul}
\sethlcolor{yellow}
\hypersetup{
	colorlinks,
	linkcolor={red!50!black},
	citecolor={red!50!black},
	urlcolor={blue!80!black}
}
\textheight=23cm \textwidth=16.5cm \oddsidemargin=0cm
\evensidemargin=0cm \topmargin=-1.75cm
\setcounter{MaxMatrixCols}{10}
\makeatletter
\g@addto@macro\@floatboxreset\centering
\makeatother
\setlength\floatsep{2\baselineskip plus 3pt minus 2pt}
\setlength\textfloatsep{2\baselineskip plus 3pt minus 2pt}
\setlength\intextsep{2\baselineskip plus 3pt minus 2 pt}
\newcommand*{\MyIndent}{\hspace*{0.5cm}}%
\usepackage[normalem]{ulem}
\newtheorem{theorem}{Proposition}
\newtheorem{definition}{Definition}
\newtheorem{proposition}{Proposition}
\newtheorem{corollary}{Corollary}
\newtheorem{example}{Example}
\newtheorem{lemma}{Lemma}
\newtheorem{remark}{Remark}
\newtheorem{assumption}{Assumptions}
\newtheorem{hypothesis}{Hypothesis}


% Uncomment the next line to use the harvard package with bibtex
%\usepackage[abbr]{harvard}

% This command determines the leading (vertical space between lines) in draft mode
% with 1.5 corresponding to "double" spacing.
%\draftSpacing{1.5}
%\newlength\TableWidth
\usepackage{array}
\newcolumntype{H}{>{\setbox0=\hbox\bgroup}c<{\egroup}@{}}
\newcommand{\Figures}{Figures/}
\newcommand{\Tables}{Tables/}
\usepackage{pifont}
\newcommand{\cmark}{\ding{51}}%
\newcommand{\xmark}{\ding{55}}%

\title{\textbf{Response to Referee 1\\ Quantitative Economics MS 2442 \\``Welfare and Spending Effects of \\ Consumption Stimulus Policies''}}
\author{Christopher D. Carroll, Edmund Crawley, William Du, \\ Ivan Frankovic, and H\aa kon Tretvoll}
\date{}

\begin{document}
	\onehalfspacing
	\maketitle
	
\noindent Thank you for your thoughtful comments and suggestions on our paper ``Welfare and Spending Effects of Consumption Stimulus Policies''. They were all very useful to us in revising the paper. We hope you agree that the paper has improved. In the following, we summarize the main changes we have made based on your, the other referees', and the editor's suggestions. Thereafter, we state each of your comments in italics and provide point-by-point responses to them.

\section{Summary of Main Changes}

\begin{itemize}
	\item \textbf{The Splurge}
	\item \textbf{Welfare Measure} We have completely overhauled our section on welfare and have introduced a new measure that we think best captures the idea that we want to measure welfare gains from carrying out each policy during a recession, but give no benefit in normal times to policies that in our model would increase welfare through redistribution. 
	
	Our welfare measure weights the felicity of a household at time $t$ by the marginal utility of the same household in a counterfactual simulation in which neither the recession occurred nor the fiscal policy was implemented. This weighting scheme means that in normal times the marginal benefit to a social planner of moving a dollar of consumption from one household at one time period to another household at the same or different time period is zero. Hence, in normal times, any re-distributive policy has zero marginal benefit. However, in a recession when the average marginal utility is higher than in normal times, there can be welfare benefits to government borrowing to allow households to consume more during the recession.
	
	Our new welfare measure leads to the same qualitative conclusions as in the previous version of the paper. It improves on the previous measure in several ways:
	\begin{itemize}
		\item The new measure does not scale with the size of the fiscal stimulus. We divide by the net present value of the fiscal policy so the measure is a `bang-bang-for-the-buck' measure. As pointed out by referee 2, our previous measure was biased by the change in the size of the UI extension policy in a recession relative to normal times.
		\item Our new measure more naturally removes the bias to policies that redistribute from high to low-marginal utility households in normal times. Previously, we took away the welfare benefit in normal times of each policy. In our new measure, ANY marginal redistributive policy has no welfare benefit in normal times.
		\item One benefit of removing the splurge from our household behavior is that our welfare measure now matches with the utility function in the households' optimization problem.
	\end{itemize}
	For more details on how we treat welfare, see section \colorbox{yellow}{XX}.
	
	\item \textbf{Robustness in a General Equilibrium Model}  The main results of this paper are presented in a partial equilibrium setup with aggregate demand effects that do not arise from a general equilibrium setup. We think there are many advantages to studying the welfare and multiplier effects in this setting without embedding the model in general equilibrium.
	
	However,  we now complement our analysis with a  general equilibrium HANK model, as standard as possible, but able to capture supply-side effects that are absent from the partial equilibrium setup and to introduce a fiscal rule to balance the government budget. We find that the consumption multipliers across horizons follow the same qualitative pattern as we have in our partial equilibrium analysis.
	
	The write up of this HANK model is found in section \colorbox{yellow}{XX}
	
	
\end{itemize}

\newpage 

\section*{Main comments}
\begin{enumerate}[label=(\alph*)]
	\item \textit{\textbf{Long-run multipliers in partial equilibrium.} This is a new comment, so
		may be discarded: I wonder how interesting the long-run multipliers	in Table 5 are. Without any supply effects, as $t \rightarrow \infty$, these just converge to $1$ as all income is eventually consumed, unless I am mistaken. This is perhaps worth stating in the discussion of Figure~7 to help the interpretation.}
	
	\noindent \textbf{Response.} We agree that it would be useful to point this out, and we have added footnote 21 to our discussion of the multipliers in section 4.2: ``In the case that there is no aggregate demand effect, these multipliers converge to 1 as $t$ goes to infinity.''
	
	\item \textit{\textbf{The new general-equilibrium analysis.} What is not standard in the GE analysis is perhaps the labor agency - does this give rise to the non-zero profits from vacancy posting? More importantluy, the authors consider in this analysis the policies in steady state, not conditional	on a recession with high unemployment. I don't understand the reason for this: the method the authors use for model-solution (based on the sequence-space Jacobian following Auclert et al) limit the analysis to small shocks around the stationary distribution in the absence of aggregate shocks. But the benefit of that method is	limited here (since the authors essentially only compute the model once, so speed is not of the essence). So why not consider the same jump in unemployment at the beginning of a non-linear transition computed using \citet{bmpMITshocks}'s method? They should also be able to compute the welfare effects in this case.}
	
	\noindent \textbf{Response.} 
\end{enumerate}

\bigskip
\newpage

\section*{Additional comments}

	\begin{enumerate}[label=(\alph*)]
		\item \textit{``Furthermore, the HANK and SAM model incorporates many other confounding and confusing elements that do more to obscure than to illuminate our points.'' (Intro). This comment seems to indicate that the authors do not think their own analysis is useful.}
		
		\noindent \textbf{Response.} We agree and we have removed the offending sentence.
	\end{enumerate}

\bigskip

\section*{Additional References}
	
	\begin{itemize}
		\item Broer, Tobias, Jeppe Druedahl, Karl Harmenberg and Erik \"Oberg, ``Stimulus effects of common fiscal policies'', mimeo. 
	\end{itemize}

	\noindent \textbf{Response.} 

\bigskip

\noindent Finally, we would like to thank you again for your careful advice on our paper. We hope you find our revision satisfactory.

\bibliographystyle{econark}
\bibliography{../../../../HAFiscal.bib,response}
\end{document}
