\input{./.econtexRoot}
\documentclass[\econtexRoot/HAFiscal]{subfiles}
\onlyinsubfile{\externaldocument{\econtexRoot/HAFiscal}} % Get xrefs -- esp to apndx -- from main file; only works if main file has already been compiled

\begin{document}
	
\FloatBarrier
\hypertarget{hank_appendix}{}\par\section{Details of the HANK Model}
\notinsubfile{\label{sec:hank_appendix}}


\subsection{Households}

The household block follows closely to the main text with a few exceptions. To begin, the splurge factor is set to zero. Secondly, the level of permanent income of all newborns is equal to one. Finally, all households face the same employment to unemployment and unemployment to employment probabilities. The probabilities are calibrated to the transition probabilities of high school graduates from the main text.


\subsection{Goods Market}

There is a continuum of  monopolistically competitive intermediate good producers indexed by $j \in [0,1]$ who produce intermediate goods $Y_{jt}$ to be sold to a final good producer at price $P_{jt}$. I assume intermediate good producers consume all profits each period. Using intermediate goods $Y_{jt}$ for $j \in [0,1]$, the  final good producer produces a final good $Y_{t}$ to be sold to households at price $P_{t}$.  \\ 


\subsubsection{Final Good Producer}

A perfectly competitive final good producer purchases intermediate goods $Y_{jt}$ from intermediate good producers at price $P_{jt}$ and produces a final good $Y_{t}$ according to a CES production function. 
$$ Y_{t} = \left(\int_{0}^{1} Y_{jt}^{\frac{\epsilon_{p}-1}{\epsilon_{p}}}\, dj\right)^{\frac{\epsilon_{p}}{\epsilon_{p}-1}}$$ 

where $\epsilon_{p}$ is the elasticity of substitution. \\

Given $P_{jt}$ , the price of intermediate good $j$ ,  the final good producer maximizes his profit by solving:
$$ \max_{Y_{jt}} P_{t} \left(\int_{0}^{1} Y_{jt}^{\frac{\epsilon_{p}-1}{\epsilon_{p}}}\, dj\right)^{\frac{\epsilon_{p}}{\epsilon_{p}-1}} - \int_{0}^{1} P_{jt} Y_{jt} ,\ dj $$ 
\vspace{.2cm}

The first order condition leads to demand for good $j$
$$Y_{jt} = \left(\frac {P_{jt}}{P_{t}}\right)^{- \epsilon_{p}} Y_{t}$$

and the price index
$$P_{t} = \left(\int_{0}^{1} P_{jt}^{1-\epsilon_{p}}\,dj \right )^{\frac{1}{1-\epsilon_{p}}}$$


\subsubsection{Intermediate Good Producer}

Intermediate goods producers produce according to a production function linear in labor $L_{t}$. 
$$Y_{jt} =  Z  L_{jt}$$ 

where $Z$ is total factor productivity.
\vspace{.3cm}
  
 Each Intermediate goods producer hires labor $L_{t}$ from a labor agency at cost $h_{t}$. 
 Given the cost of labor, each Intermediate goods producer chooses $P_{jt}$ to maximize its profit facing price stickiness a la \cite{Rotemberg1982}. I assume intermediate good producers hold all profits as HANK models with sticky prices produce countercyclical profits which combined with households with high MPCs can lead to countercyclical consumption responses out of dividends. I therefore abstract from consumption behavior in response to firm profits. Intermediate goods producers maximize profit by solving:
 
$$J_{t}\left(P_{jt}\right) = \max_{\{P_{jt}\}} \left\{\frac{P_{jt}Y_{jt}}{P_{t}} - h_{t} L_{jt} -  \frac{\varphi}{2}\left( \frac{P_{jt} - P_{jt-1}}{P_{jt-1}} \right)^{2} Y_{t}  + J_{t+1}\left(P_{jt+1}\right) \right\}$$ 

The problem can be rewritten as the standard New Keynesian maximization problem:

$$\max_{\{P_{jt}\}} \mathrm{E}_{t}\left[\sum_{s=0}^{\infty}  M_{t,t+s} \left( \left( \frac{P_{jt+s}}{P_{t+s}} - MC_{t+s}\right)Y_{jt+s} -  \frac{\varphi}{2}\left( \frac{P_{jt+s}}{P_{jt+s-1}} - 1\right)^{2} Y_{t+s} \right)\right]$$ 


where $MC_{t} = \frac{h_{t}}{Z_{t}}  $ \\



Given all firms face the same adjustment costs, there exists a symmetric equilibrium where all firms choose the same price with $P_{jt} =P_{t}$ and $Y_{jt} =Y_{t}$.\\ 

The resulting Phillips Curve is


$$ \epsilon_{p} MC_{t} = \epsilon_{p} - 1 + \varphi ( \Pi_{t} -1) \Pi_{t} - M_{t,t+1} \varphi (\Pi_{t+1} -1 ) \Pi_{t+1} \frac{Y_{t+1}}{Y_{t}}$$

where $ \Pi_{t} = \frac{P_{t}}{P_{t+1}}$. \\



\subsection{Labor market}

\subsubsection{Labor agency}

A risk neutral labor agency sells labor $N_{t}$ to intermediate good producers at cost $h_{t}$ by hiring households. To hire households, the labor agency posts vacancies $v_{t}$ that are filled with probability $\phi_{t}$.  Households search is random. Following \cite{Bardoczy2022}, I assume the labor agency cannot observe the labor productivity of individual households. Instead, the labor agency can only observe the average productivity of all employed workers which is always equal to one. 

$$J_{t}(N_{t-1})  = \max_{N_{t},v_{t}} \{( h_{t} - w_{t}) N_{t}- \kappa v_{t} + \mathrm{E_{t}}\left[ \frac{J_{t+1}(N_{t})}{1 + r^{a}_{t}}\right]\}$$

s.t.
$$ N_{t} = (1-\omega)N_{t-1} + \phi_{t} v_{t}$$ 

The resulting job creation curve is:
$$ \frac{\kappa}{\phi_{t}}  = (h_{t} - w_{t})+  (1-\omega)\mathrm{E_{t}}\left[   \frac{\kappa}{(1+r^{a}_{t}) \phi_{t+1}} \right]   $$


\subsubsection{Matching}

Household and labor agency matching follows a Cobb Douglas matching function:

$$m_{t} = \chi e_{t}^{\alpha} v_{t}^{1-\alpha}$$ 

where $m_{t}$ is the mass of matches, $ e_{t} $ is the mass of job searchers, and $\chi$ a matching efficiency parameter.\\

The vacancy filling probability $\phi_{t}$, job finding probabilities $\eta_{t}$ evolve according to:

$$\eta_{t} = \chi \Theta_{it}^{1-\alpha} $$
%$$\eta_{t}(X) = \chi q(X) \Theta_{it}^{1-\alpha} $$
$$ \phi_{t} = \chi \Theta_{t}^{-\alpha} $$ 
\vspace{.1cm}

where $\Theta_{t} = \frac{v_{t}}{e_{t}}$ is labor market tightness.



\subsection{Wage Determination }


Similar to \cite{Gornemann2021} and \cite{Blanchard2010}, I assume the real wage follows the rule :

$$log\left(\frac{w_{t}}{w_{ss}}\right)  = \phi_w log\left( \frac{ w_{t-1}}{ w_{ss}} \right) +   (1 - \phi_w) log\left( \frac{N_{t}}{N_{ss}}\right)$$
\vspace{.2cm}

where $\phi_w$ dictates the extent real wages are rigid. 



\subsection{Fiscal Policy}

The government issues long term bonds $B_{t}$ at price $q^{b}_{t}$ in period $t$ that pays $\delta^{s}$ in period $t+s+1$ for $s = 0,1,2, ..$ \\

The bond price satisfies the no arbitrage condition:

$$q^{b}_{t} = \frac{ 1  + \delta \mathrm{E}_{t}[q^{b}_{t+1}]}{1+r^{a}_{t}}$$ 


The government finances its expenditures with debt and taxes. 

$$ (1 + \delta q^{b}_{t})B_{t-1} + G_{t}  + S_{t} = \tau_{t} w_{t} N_{t}+ q^{b}_{t}B_{t}$$
\vspace{.4cm}

where $ S_{t}$ are payments for unemployment insurance and other transfers. \\

Following \cite{AuclertMicroJumpsMacroHumps}, the tax rate adjusts to stabilize the debt to GDP ratio:\\
$$ \tau_{t} - \tau_{ss} = \phi_{B} q^{b}_{ss} \frac{B_{t-1} - B_{ss} }{Y_{ss}}$$

\vspace{.4cm}
where $\phi_{B}$ governs the speed of adjustment. 



\subsection{Monetary Policy}


The central bank follows the Taylor rule: 

$$i_{t} = r^{*} +\phi_{\pi} \pi_{t}  + \phi_{Y} (Y_{t} -  Y_{ss})$$ 


where $\phi_{\pi}$ and $\phi_{Y} $ are the Taylor rule coefficient for inflation and output, respectively.  $r^{*}$ is the steady state interest rate, $Y_{ss}$ is the steady state level of output. \\

\subsection{Equilibrium}


An equilibrium in this economy is a sequence of: \\

- Policy Functions $\left( c_{it}(m) \right )_{t=0}^{\infty}$ normalized by permanent income \\

- Prices $ \left(r_{t},  r^{a}_{t+1}, i_{t}, q^{b}_{t},  w_{t}, h_{t} , \pi_{t} , \tau_{t} \right) _{t=0}^{\infty}$\\

- Aggregates $ \left(C_{t}, Y_{t} , N_{t},   \Theta_{t},  B_{t} , A_{t}  \right)_{t=0}^{\infty}$\\

Such that: \\

$ \left(  c_{it}(m)\right)_{t=0}^{\infty}$  solves the household's maximization problem given $  \left( w_{t}, \eta_{t},  r^{a}_{t} , \tau_{t} \right)_{t=0}^{\infty}$.\\

The final goods producer and intermediate goods producers maximize their objective function. \\

The nominal interest rate is set according to the central bank's Taylor rule. \\

The tax rate is determined by the fiscal rule and the government budget constraint holds. \\

The value of assets is equal to the value of government bonds.:
 $$ A_t =  q^{b}_{t}B_{t}  $$

 The goods market clears\footnote{Note if profits were not held by firms then the goods market condition would be $ C_{t}  + G_{t}  = Y_{t} -  \kappa v_{t} - \frac{\varphi}{2}\left( \frac{P_{t}}{P_{t-1}} - 1\right)^{2} Y_{t}  $.  In particular, since firm profits are $D_{t} = Y_{t} -  w_{t} N_{t}  - \kappa v_{t} - \frac{\varphi}{2}\left( \frac{P_{t}}{P_{t-1}} - 1\right)^{2} Y_{t} $, then the goods market condition would become $ C_{t}  + G_{t}  =w_{t} N_{t}  + D_{t} = Y_{t} -  \kappa v_{t} - \frac{\varphi}{2}\left( \frac{P_{t}}{P_{t-1}} - 1\right)^{2} Y_{t}  $. }: 
$$ C_{t}  = w_{t} N_{t}  + G_{t} $$

 where $C_{t} \equiv  \int_{0}^{1} \textbf{p}_{it} c_{it}\, di $ 

The labor demand of intermediate good producers equals labor supply of labor agency:
$$ L_{t} =  N_{t}$$ 





%\input{\TableDir/Calibration.tex}


\begin{table}
\begin{center}
\renewcommand{\arraystretch}{1.6}
\caption{Calibration}\label{table:Calibration}
\makebox[\textwidth]{\begin{tabular}{c c c l}
Description & Parameter & Value & Source/Target \\ \hline
Job Separation Rate & $\omega$ & 0.1 & JOLTS \\
Job Finding Probability & $\eta_{t}$ & .67 & EU probability \\
Elasticity of Substitution & $\epsilon_{p}$ & 6 & Standard \\
Price Adjustment Costs & $\varphi$ & 96.9 & \cite{Ravn2017} \\
Vacancy Filling Rate & $\phi$ & 0.71 & \cite{DenHaan2000} \\
Matching Elasticity & $\alpha$ & 0.65 & \cite{Ravn2017} \\
Real Wage Rigidity parameter & $\phi_{w}$ & 0.837 & \cite{Gornemann2021} \\
Vacancy Cost & $\kappa$ & 0.056 & $\frac{\kappa}{w\phi} = 0.071$ \\
Government Spending & $G$ & 0.38 & Gov. budget constraint \\
Decay rate of Government Coupons & $\delta$ & 0.95 & 5 Year Maturity of Debt \\
Taylor Rule Inflation Coefficient & $\phi_{\pi}$ & 1.5 & Standard \\
Response of Tax Rate to Debt & $\phi_{b}$ & 0.015 & \cite{Auclert2020} \\ \hline
\end{tabular}}
\end{center}
\end{table}

\section{Calibration of Non-Household Blocks }

The job separation rate is set to 0.1 in line with JOLTS. I set the job finding probability for the employed $\eta_{t}$ is set to 0.67. The quarterly vacancy filling rate is 0.71 as in \cite{DenHaan2000}. The matching elasticity is 0.65 following \cite{Ravn2017} and the vacancy cost is set to 7\% of the real wage as in \cite{Christiano2016}.\footnote{The range of plausible values lie between $4\%$ and $14\%$ \cite{Silva2009}} The elasticity of substitution is set to 6. The price adjustment cost parameter is set to 96.9 as in \cite{Ravn2017}. The tax rate is set to 0.3 and government spending is set to clear the government budget constraint. The speed of fiscal adjustment, $\phi_{b}$, is set to 0.015, the lower bound of the estimates in  \cite{Auclert2020}.\footnote{The speed of adjustment parameter is set to the lower bound to ensure that the policies evaluated in the HANK and SAM model are almost entirely deficit financed.} Furthermore, the decay rate of government coupons is set to $\delta = 0.95$ to match a maturity of 5 years\footnote{The duration of bonds in the model is $\frac{(1+r)^4}{(1+r)^4 - \delta}$ }.


\ifthenelse{\boolean{Web}}{}{
\end{document} \endinput
}

