\documentclass[../HAFiscal]{subfiles}
\begin{document}

\section{Welfare analysis}

In the previous section we analyzed the three models for their effects on spending. In this section we look at the welfare implications of each stimulus policy. There is no one right way in which to aggregate welfare in a model with individual utility functions. Our approach captures three ideas: (1) The utility of each consumer is valued equally by the social planner; (2) Each policy has no social benefit when implemented outside of a recession; and (3) Utility is gained from `splurge' spending in the same way as other spending. The first of these would suggest a simple aggregation of individual wel...

Let  $\mathcal{W}(\text{policy},AD,Rec)$ be the aggregated utility function:
\begin{align}
	\mathcal{W}(\text{policy},AD,Rec) =\frac{1}{N}\sum_{i=1}^{N} \sum_{t=0}^{\infty} D^t  u(c_{it,\text{policy},AD,Rec}) 
\end{align}
where $\text{policy}\in \{\text{None, Extended UI, Stimulus Checks, Payroll Tax Cut}\}$ is the stimulus policy followed,  $AD\in\{1,0\}$ is and indicator for whether the aggregate demand effects occur while the recession is active, and  $Rec\in\{1,0\}$ is an indicator for whether the policy coincides with the start of a recession or is implemented in non-recessionary times. $c_{it,\text{policy},AD,Rec}$ are the consumption paths (including the splurge) for each consumer $i$ in each scenario. $D$ is the social planner's discount factor that we will set to be equal to the real interest rate $R$. $N$ is the number of consumers simulated.

We use the steady-state baseline as a way to convert from welfare units to consumption units. Using this baseline, we define the marginal increase in welfare that occurs when every consumer increases their consumption proportionally to their baseline consumption as:\footnote{Note that with log utility, $\mathcal{W}^c =\frac{1}{N}\sum_{i=1}^{N} \sum_{t=0}^{\infty} D^t = \frac{1}{1-D}$}
\begin{align}
	\mathcal{W}^c =\frac{1}{N}\sum_{i=1}^{N} \sum_{t=0}^{\infty} D^t c_{it,\text{None},0,0} u'(c_{it,\text{None},0,0}) .
\end{align}
With this definition we consider, in steady-state consumption units $\mathcal{W}^c$, the increase in welfare induced by a policy: $\frac{\mathcal{W}(\text{policy},AD,Rec)-\mathcal{W}(AD,Rec)}{\mathcal{W}^c}$. The present value of the fiscal payments made by the government for each policy is $PV(\text{policy},Rec)$.\footnote{For the stimulus check and extended unemployment insurance the payments made by the government are clearly defined and do not depend on aggregate demand effects. For the payroll tax cut, we define the payments as the difference between the take-home pay with and without the tax cut, but ignoring any aggregate demand effects. Aggregate demand effects would increase the value of the tax cut, because incomes would rise, but in fact increase rather than decrease the tax receipts of the government.} We subtract the fiscal cost of each policy in steady-state consumption units:  $\frac{PV(\text{policy},Rec)}{{P}^c}$ where ${P}^c$, the marginal cost of increasing every consumer's steady-state consumption proportionally, is given by:
\begin{align}
	\mathcal{P}^c = \frac{1}{N}\sum_{i=1}^{N} \sum_{t=0}^{\infty} R^{-t} c_{it,\text{None},0,0} .
\end{align}
Finally, we normalize the welfare benefit by subtracting the welfare effect of the policy in non-recessionary times. This can be thought to encompass both the preferences of society not to redistribute and the negative incentive effects of redistribution in normal times. Our final welfare measure, expressed in units of steady-state consumption, is:
\begin{align}
	\mathcal{C}(\text{policy},AD,Rec) &= \bigg(\frac{\mathcal{W}(\text{policy},AD,Rec)-\mathcal{W}(AD,Rec)}{\mathcal{W}^c} - \frac{PV(\text{policy},Rec)}{\mathcal{P}^c} \bigg)\\ \nonumber
	& \qquad -
	\bigg(\frac{\mathcal{W}(\text{policy}) - \mathcal{W}(\text{base})}{\mathcal{W}^c} - \frac{PV(\text{policy})}{\mathcal{P}^c} \bigg) \label{welfare_defn}
\end{align}
Table \ref{welfare} shows the welfare benefits of each policy as defined by equation \eqref{welfare_defn}. The stimulus check and payroll tax cut policies have been adjusted to be the same fiscal size as the unemployment insurance extension.\footnote{As shown in appendix \ref{app:CheckMultiplier}, the multiplier of the check stimulus is insensitive to its size. Hence, the downscaling of the stimulus check does not alter the impact on consumption per unit of policy expenditure.} Without aggregate demand effects (the first row of the table), the payroll tax cut has extremely limited welfare benefits. This is because the payroll tax cut goes to consumers who remain employed and therefore does not directly affect the unemployed consumers who are the most hit by the recession. However, employed consumers do reduce their consumption at the onset of the recession due to the increased unemployment risk, so the tax cut helps them more than in non-recessionary times.  Similarly, the stimulus check has limited benefit as it mostly goes to employed consumers although it has the benefit over the payroll tax cut of also reaching the unemployed. However, the extended unemployment insurance policy is the clear `bang for the buck' winner as all the payments go to unemployed households who are likely to have significantly higher marginal utility for consumption than in non-recessionary times.

The second row of the table shows the welfare benefits in the version of the model with an aggregate demand multiplier during the recession. The payroll tax cut now has a noticeable benefit as some of the tax cut gets spent during the recession resulting in higher incomes for all consumers. However, the tax cut is received over a period of two years, and much of this may be after the recession---and hence the aggregate demand effect---is over. Furthermore, because the payroll tax cut goes only to employed consumers who have relatively lower MPCs, the spending out of this stimulus will be further delayed possibly beyond the period of the recession. By contrast, the stimulus check is received in the first period of the recession and goes to both employed and unemployed consumers. The earlier arrival and higher MPCs of the stimulus check recipients means more of the stimulus is spent during the recession leading to greater aggregate demand effects, higher income, and higher welfare. The extended unemployment insurance arrives, on average, slightly later than the stimulus check. However, the recipients, who have been unemployed for at least six months, spend the extra benefits relatively quickly resulting in significant aggregate demand effects during the recession. In contrast to the payroll tax cut, extended unemployment insurance has the benefit of automatically reducing if the recession ends early and less consumers are eligible for the benefit.

\begin{table}[ht] 
	\center
	\input Code/HA-Models/FromPandemicCode/Tables/welfare4.tex
	\caption{Consumption Equivalent Welfare Gains in Basis Points }
	\label{welfare}
\end{table}


\end{document}	
