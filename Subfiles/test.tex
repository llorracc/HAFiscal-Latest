% -*- mode: LaTeX; TeX-PDF-mode: t; -*- # Tell emacs the file type (for syntax)
\newcommand{\econtexRoot}{.}

\documentclass[\econtexRoot/HAFiscal]{subfiles}
\onlyinsubfile{\externaldocument{\econtexRoot/HAFiscal}} % Get xrefs -- esp to apndx -- from main file; only works if main file has already been compiled

\begin{document}

\hypertarget{introduction}{}\par\section{Introduction}\notinsubfile{\label{sec:intro}}
\setcounter{page}{0}\pagenumbering{arabic}


Fiscal policies that aim to boost consumer spending in recessions have been tried in many countries in recent decades.  The nature of such policies has varied widely, perhaps because traditional macroeconomic models have not provided plausible guidance about which policies are likely to be most effective---either in reducing misery (a `welfare metric') or in increasing output (a `GDP metric').

But a new generation of macro models has shown that when microeconomic heterogeneity across consumer circumstances (wealth; income; education) is taken into account, the consequences of an income shock for consumer spending depend on a measurable object: the intertemporal marginal propensity to consume (IMPC) introduced in \cite{auclert2018IKC}.  The IMPC extends the notion of marginal propensity to consume (MPC) to account for the speed at which households spend.  Fortuitously, new sources of microeconomic data, particularly from Scandinavian national registries, have recently allowed the first reasonably credible measurements of the IMPC (\cite{fagereng_mpc_2021}).

\begin{equation}
  a = b
  \end{equation}

\onlyinsubfile{\LaTeXInputs/bibliography_blend}
\onlyinsubfile{% Allows two (optional) supplements to hard-wired \texname.bib bibfile:
% global.bib is a default bibfile that supplies anything missing elsewhere
% Add-Refs.bib is an override bibfile that supplants anything in \texfile.bib or global.bib
\provideboolean{AddRefsExists}
\provideboolean{globalExists}
\provideboolean{BothExist}
\provideboolean{NeitherExists}
\setboolean{BothExist}{true}
\setboolean{NeitherExists}{true}

\IfFileExists{\econtexRoot/Add-Refs.bib}{
  % then
  \typeout{References in Add-Refs.bib will take precedence over those elsewhere}
  \setboolean{AddRefsExists}{true}
  \setboolean{NeitherExists}{false} % Default is true
}{
  % else
  \setboolean{AddRefsExists}{false} % No added refs exist so defaults will be used
  \setboolean{BothExist}{false}     % Default is that Add-Refs and global.bib both exist
}

% Deal with case where global.bib is found by kpsewhich
\IfFileExists{/usr/local/texlive/texmf-local/bibtex/bib/global.bib}{
  % then
  \typeout{References in default global global.bib will be used for items not found elsewhere}
  \setboolean{globalExists}{true}
  \setboolean{NeitherExists}{false}
}{
  % else
  \typeout{Found no global database file}
  \setboolean{globalExists}{false}
  \setboolean{BothExist}{false}
}

\ifthenelse{\boolean{showPageHead}}{ %then
  \clearpairofpagestyles % No header for references pages
  }{} % No head has been set to clear

\ifthenelse{\boolean{BothExist}}{
  % then use both
  \typeout{bibliography{\econtexRoot/Add-Refs,\econtexRoot/\texname,global}}
  \bibliography{\econtexRoot/Add-Refs,\econtexRoot/\texname,global}
  % else both do not exist
}{ % maybe neither does?
  \ifthenelse{\boolean{NeitherExists}}{
    \typeout{bibliography{\texname}}
    \bibliography{\texname}}{
    % no -- at least one exists
    \ifthenelse{\boolean{AddRefsExists}}{
      \typeout{bibliography{\econtexRoot/Add-Refs,\econtexRoot/\texname}}
      \bibliography{\econtexRoot/Add-Refs,\econtexRoot/\texname}}{
      \typeout{bibliography{\econtexRoot/\texname,global}}
      \bibliography{        \econtexRoot/\texname,global}}
  } % end of picking the one that exists
} % end of testing whether neither exists
}

\ifthenelse{\boolean{Web}}{}{
%  \onlyinsubfile{\captionsetup[figure]{list=no}}
%  \onlyinsubfile{\captionsetup[table]{list=no}}
  \end{document} \endinput
}

