% -*- mode: LaTeX; TeX-PDF-mode: t; -*- # Tell emacs the file type (for syntax)
\input{./.econtexRoot}
\documentclass[\econtexRoot/HAFiscal]{subfiles}
\onlyinsubfile{\externaldocument{\econtexRoot/HAFiscal}} % Get xrefs -- esp to apndx -- from main file; only works if main file has already been compiled

\begin{document}

\hypertarget{related-literature}{}\par\subsection{Related literature}
\notinsubfile{\label{sec:lit}}

Several papers have looked at fiscal policies that have been implemented in the U.S.\ under the lens of a structural model. \cite{coenen2012effects} analyses the effects of different fiscal policies using seven different models. The models are variants of two-agent heterogeneous agent models and make no attempt to match the full distribution of liquid wealth as we do in this paper. We also attempt to match the microdata on household consumption behavior, much of which has come more recently.  More closely aligned to the methodology of our paper are \cite{mckay2016role}, \cite{mckay2021optimal}, and \cite{phan2024welfare} which look at the role of automatic stabilizers. By contrast, we consider discretionary policies that have been invoked after a recession has begun. \cite{bayercoronavirus} also study fiscal policies implemented after a large shock, in their case the COVID-19 pandemic. They find that targeted stimulus through an increase in unemployment benefits has a much larger multiplier than an untargeted policy. In contrast, we find that untargeted stimulus checks have slightly higher multiplier effects when compared with a targeted policy extending eligibility for unemployment insurance. Our results derive from the fact that---as in the data---even high liquid wealth consumers have relatively high MPCs in our model. 

Another related paper is \cite{broer2025stimulus} that analyze the output response to different fiscal policies in a HANK and SAM model similar to the one we present in our robustness exercise. They consider policies involving transfers to firms as well as households, but implement the policies in a steady state rather than a recession and only consider output responses. Unlike us, they do not calibrate their model to match the wealth distribution and the iMPCs, and they do not evaluate the policies using a welfare metric. 

This paper is also closely related to the empirical literature that aims to estimate the effect of transitory income shocks and stimulus payments. We particularly focus on \cite{fagereng_mpc_2021}, who use Norwegian administrative panel data with sizable lottery wins to estimate the MPC out of transitory income in that year, as well as the pattern of expenditure in the following years. A closely related paper is \cite{kotsogiannisMPCs}, who use data from a Greek lottery, and find that the induced extra monthly spending in the first three months is triple the induced extra spending in the remaining observed months. We build a model that is consistent with the patterns they identify which includes an excess initial MPC. Examples of the literature that followed the Great Recession in 2008 are \cite{parker2013consumer} and \cite{broda2014economic}. These papers exploit the effectively random timing of the distribution of stimulus payments and identify a substantial consumption response. The results indicate an MPC that is difficult to reconcile with representative agent models.

Thus, the paper relates to the literature presenting HA models that aim to be consistent with the evidence from the micro-data. An example is \cite{kaplan2014model}, who build a model where agents save in both liquid and illiquid assets. The model yields a substantial consumption response to a stimulus payment, since MPCs are high both for constrained, low-wealth households and for households with substantial net worth that is mainly invested in the illiquid asset (the ``wealthy hand-to-mouth''). \cite{carroll2020modeling} present an HA model that is similar in many respects to the one we study. Their focus is on predicting the consumption response to the 2020 U.S.\ CARES Act that contains both an extension of unemployment benefits and a stimulus check. However, neither of these papers attempts to evaluate and rank the effectiveness of different stimulus policies, as we do.

\cite{kaplanMPC2022} discuss different mechanisms used in HA models to obtain a high MPC and the tension between that and fitting the distribution of aggregate wealth. We use one of the mechanisms they consider, \textit{ex-ante} heterogeneity in discount factors, and build a model that delivers both high average MPCs and a distribution of liquid wealth consistent with the data. The model allows for splurge consumption and thus also delivers substantial MPCs for high-liquid-wealth households.\footnote{In addition to \cite{fagereng_mpc_2021}, \cite{boehm2025fivefacts}, \cite{graham2024mental}, \cite{crawley2023MicroMacro}, and \cite{kueng2018excess} among others all provide strong evidence of high MPCs for high-liquid-wealth households. By contrast, the ``infrequent consumption good'' by \cite{melcangiStock} aims at accounting for high saving rates among high-income households during normal times and high consumption during episodes where the infrequent consumption good becomes available (such as high-end health care, education expenses or bequests). The high consumption is thus not triggered by a transitory income shock but by rare consumption opportunities.} This helps the model match not only the initial MPC, but also the propensity to spend out of a windfall for several periods after it is obtained.

We include splurge consumption as a model device that lets the model match the empirical evidence, but in the literature there are several attempts at explaining such behavior. One possibility is that the burst of initial spending is rationalizable if the spending is on durables \citep{bcShocksStocks}.  \cite{mankiw:durgoods} showed that in the frictionless case, spending on durable goods should be vastly more responsive to a permanent income shock than spending on nondurables. It seems plausible that a model with a large number of goods that are durable at, say, the quarterly or annual frequency could explain the `excess initial MPC' as actually reflecting a rational marginal propensity to Expend (MPX).\footnote{The NIPA accounts treat as `durable' those goods whose expected lifetime is 3 years or more, but at the annual frequency many more things are arguably durable -- for example, \cite{bdTimeSeriesC} mention clothes and shoes, and \cite{hkpMemorable} argue that many services are durable at the annual frequency, which explains why people take vacations once a year.} 

Alternatively, \cite{Lian2023-ca} shows how deviations from an optimal consumption policy function in the future can result in high MPCs today. Several behavioral biases can account for such mistakes. \cite{BoutrosWindfall} and \cite{ilutEconomic} present models with with bounded rationality and costly re-optimization, while \cite{indarte2024explains} and \cite{lmmPresentBias} attribute the excess initial MPX to a form of ``present bias'' in which people have strongly time inconsistent preferences. \cite{laibson2022simple} combine both present bias and durables expenditures in a simple model.  A back-of-the-envelope calculation yields a rough estimate that the ratio of initial spending on durables to the spending that would occur if all spending were nondurable is roughly three to one (not far from the ratio estimated in the Greek lottery episode studied by \cite{kotsogiannisMPCs}).

In our model, consumers do not adjust their labor supply in response to the stimulus policies.  Our assumption is broadly consistent with the empirical findings in \cite{ganong2022spending} and \cite{chodorow2016limited}. However, the literature is conflicted on this subject and \cite{hagedorn2017impact} and \cite{hagedorn2019unemployment} find that extensions of unemployment insurance affect both search decisions and vacancy creation leading to a rise in unemployment. \cite{kekre2022unemp}, on the other hand, evaluates the effect of extending unemployment insurance in the period from 2008 to 2014. He finds that this extension raised aggregate demand and implied a lower unemployment rate than without the policy. However, he does not attempt to compare the stimulus effects of extending unemployment insurance with other policies. Finally, \cite{cohenDisemployment} employ a meta-analysis of the literature on how unemployment benefits impact unemployment duration, and they find that the effects are modest. 

One criterion to rank policies is the extent to which spending is ``multiplied,'' and our paper therefore relates to the vast literature discussing the size and timing of any multiplier. Our focus is on policies implemented in the aftermath of the Great Recession, a period when monetary policy was essentially fixed at the zero lower bound (ZLB). We therefore do not consider monetary policy responses to the policies we evaluate in our primary analysis, and our work thus relates to papers such as \cite{christiano2011government} and \cite{eggertsson2011fiscal}, who argue that fiscal multipliers are higher in such circumstances. \cite{hagedorn2019fiscal} present an HA model with both incomplete markets and nominal rigidities to evaluate the size of the fiscal multiplier and also find that it is higher when monetary policy is constrained. Unlike us, they focus on government spending instead of transfers and are interested in different options for financing that spending. \cite{broer2023fiscalmultipliers} also focus on fiscal multipliers for government spending and show how they differ in representative agent and HA models with different sources of nominal rigidities. \cite{ramey2018government} investigate empirically whether there is support for the model-based results that fiscal multipliers are higher in certain states. While they find evidence that multipliers are higher when there is slack in the economy or the ZLB binds, the multipliers they find are still below one in most specifications. In any case, we condition on policies being implemented in a recession---when, this literature argues, multipliers are higher---but it is not crucial for our purposes whether the multipliers are greater than one or not. We are concerned with relative multipliers, and the multiplier is only one of the two criteria we use to rank policies. 

The second criterion to rank policies is our measure of welfare. Thus, the paper relates to the recent literature on welfare comparisons in HA models. Both \cite{bhandari2021efficiency} and \cite{davila2022welfare} introduce ways of decomposing welfare effects. In the former case, these are aggregate efficiency, redistribution and insurance, while the latter further decomposes the insurance part into intra- and intertemporal components. These papers are related to ours, but we do not decompose the welfare effects. Regardless of decomposition, we want to (1) use a welfare measure as an additional way of ranking policies and (2) introduce a measure that abstracts from any incentive for a planner to redistribute in the steady state (or ``normal'' times).

\hypertarget{organization}{}\par\subsection{Organization}
\notinsubfile{\label{sec:org}}

The paper is organized as follows. Section~\ref{sec:model} presents our baseline partial equilibrium model of households' consumption and saving problem as well as how we model a recession and the potential response in terms of three different consumption stimulus policies. Section~\ref{sec:parameters} describes the steps we take to parameterize the model and discusses the implications for some moments that we do not target. In section~\ref{sec:comparing} we compare the three policies implemented in a recession both in terms of their multipliers and in terms of a welfare measure that we introduce. Section~\ref{sec:hank} presents a general equilibrium HANK and SAM model where we compare the multipliers of the same three policies to the partial equilibrium results. Section~\ref{sec:conclusion} concludes, and, finally, the appendix shows results from a version of the model without splurge consumption and provides more details of the HANK and SAM model discussed in Section~\ref{sec:hank}. 


\onlyinsubfile{\bibliography{\bibfilesfound}}


%\onlyinsubfile{bibliography_blend}
%\onlyinsubfile{\input{bibliography_blend}}

\ifthenelse{\boolean{Web}}{}{
%  \onlyinsubfile{\captionsetup[figure]{list=no}}
%  \onlyinsubfile{\captionsetup[table]{list=no}}
  \end{document} \endinput
}

