\documentclass[../HAFiscal]{subfiles}
\begin{document}
Fiscal policies that aim to boost consumer spending in recessions have been tried in many countries in recent years.  The nature of these policies has varied widely, perhaps because traditional macroeconomic models have not provided plausible guidance about which policies are likely to be most effective.

But a new generation of macro models has shown that when microeconomic heterogeneity in consumer circumstances (wealth; income) is taken into account, the consequences of an income shock for consumer spending depend on a measurable object: the `intertemporal marginal propensity to consume' (`IMPC').  Fortuitously, new sources of microeconomic data, particularly from Scandinavian national registries, have recently allowed the first reasonably credible measurements of the IMPC (\cite{fagering_mpc_2021}).

<<<<<<< HEAD
This combination of developments makes it possible, really for the first time, to conduct quantitatively credible structural analyses of the likely effectiveness of alternative consumption stimulus choices.
=======
As recent work of \cite{auclert2018IKC} has emphasized, the consequences of an income shock depend on the `intertemporal marginal propensity to consume' (the `IMPC').  We construct an HA model that matches the IMPC as identified in \cite{fagereng_mpc_2021} using Norwegian registry data, and at the same time matches the distribution of liquid assets in the population.  In principle, the resulting structural model could be used to evaluate any number of alternative stimulative policies.  Here, we apply it to evaluate three policies that have been used in recent recessions in the U.S. and elsewhere: an extension of unemployment insurance (UI) benefits; a means-tested stimulus check, and a payroll tax cut.  We begin by comparing the policies' effects on aggregate consumption expenditures.  But because the model's outcomes reflect the behavior of utility maximizing consumers, we can also evaluate policies directly in terms of the impact on consumers' utility. The principal difference between the two metrics is that what matters for the degree of multiplication is how much of the extra spending occurs during the period when the multiplier exists (or is powerful), while the utility metric also depends on who is doing the extra spending (because people with a high marginal propensity to consume are also likely to have a high marginal utility of consumption).  
    
In the case of the policies compared here, the key characteristics are how well targeted the policy is and the timing of the spending it induces. We assume that an advantage of the stimulus checks is that they are distributed immediately upon commencement of the recesssion, at which point any multiplier is fully in force; this early timing increases the likelihood that any extra spending will occur at a point where it is multiplied. The UI extension, on the other hand, is targeted to recipients with high MPCs, but some of the spending attributable to the UI extension will occur after the recession is over. The recipients' high MPC implies that the utility consequences of the UI policy for them will still be considerable, but their post-recession spending will not have a multiplier effect.  See our section~\ref{sec:comparing} for quantitative measures of how these points play out.
>>>>>>> bdd75cb0b3a84cf519401155286c56d02a20b3b1

Here, we construct an HA model calibrated to match both the \cite{fagereng_mpc_2021}-measured IMPC and a measure of the distribution of liquid assets across consumers.  The only substantial innovation in our model (relative to the existing HA-macro literature) is introduced to allow our model match a substantial body of evidence, from \cite{fagereng_mpc_2021} and elsewhere,\footnote{\cite{parker_etal}, [ganong], etc} that the immediate impact of a change in income is to induce a disproportionate change in spending immediately.  (`Disproportionate' in the sense that a model of optimal nondurable consumption spending does not match the measured steep falloff in spending between the first period after the income shock, and subsequent periods.)\footnote{A popular explanation for this first-period excess spending is that it may reflect spending on `small durables' rather than pure nondurables (cf. [citations]).}  We capture this by assuming that consumers spend a fixed fraction of their labor income each period, which we call the `splurge' factor. This spending occurs regardless of their current wealth and fits with the empirical evidence that even high liquid-wealth households have high initial MPCs.[citations] By contrast, in a standard one- or two-asset buffer-stock model, high liquid wealth households smooth their consumption through transitory shocks and exhibit low MPCs.\footnote{The splurge is also consistent with evidence, from \cite{ganongConsumer2019}, that spending drops sharply following the large and predictable drop in income following the exhaustion of unemployment benefits.}

The resulting structural model could be used to evaluate almost any plausible consumption stimulus policy.  Here, we use it to evaluate three policies that have been implemented in recent recessions in the U.S.\ (and elsewhere): an extension of unemployment insurance (UI) benefits; a means-tested stimulus check; and a payroll tax cut.

Our first metric of policy effectiveness is `spending bang for the buck': For a dollar of spending on a particular policy, by how much (and when) is spending affected.  Timing can matter because the size of any `consumption multiplier' likely depends on the economic conditions that prevail when the extra spending occurs.  Our strategy to illuminate this point is twofold.  First, we calculate the policy-induced spending dynamics in an economy with no multiplier.  Then, we make the assumption that there is no multiplication (at all) for spending that occurs after our simulated recession is over.  A less stark assumption (e.g., the degree of multiplicion depends on the distance of the economy from its steady state) would perhaps be more realistic but also much harder to assess clearly.  (Our on-off multiplier is possible because we follow \cite{kmpHandbook2016}'s approach to modeling the multiplication phenomenon).

Because our model's outcomes reflect the behavior of utility maximizing consumers, we can calculate another, possibly more interesting, measure of the effectiveness of alternative policies:  Their effect on consumers' welfare.  Even without multiplication, a utility-based metric can justify countercyclical policy because the larger idiosyncratic shocks to income that occur during a recession may justify a greater-than-normal degree of social insurance.

The principal difference between the two metrics is that what matters for the degree of multiplication is how much of the policy-induced extra spending occurs during the recession (when the multiplier matters), while effectiveness in the utility metric also depends on who is doing the extra spending (because different recipients have very different marginal utilities).  

In the case of the policies compared here, an advantage of the stimulus checks (as we model them) is that they are distributed immediately upon commencement of the recesssion, at which point the multiplier is fully in force; most extra spending will therefore occur at a point where it is multiplied.  But some of the spending attributable to the unemployment insurance extension will occur after the recession is over.  The fact that UI recipients have a high MPC implies that the utility consequences of the UI policy for them will still be considerable, even if their post-recession spending does not get multiplied.  (Section~\ref{sec:comparing}).

We use the model to aggregate consumer utility into a social welfare function. Because low-income consumers have high marginal utility, a standard aggregated welfare function would favor re-distributive policies even in the absence of a recession. To avoid weighting our analysis toward re-distributive policies, we normalize our social welfare criteria such that each policy, implemented in non-recessionary times, has zero welfare benefit to the social planner. 

Recessions are unexpected (they are `MIT shocks') and double the unemployment rate and the average length of unemployment spells. The end of the recession occurs as a Bernoulli process calibrated for an average length of recession of six quarters, leading to a return of the unemployment rate to normal levels over time. In an extension to the model, we allow for an aggregate demand multiplier effect during the recession, following the method introduced by \cite{kmpHandbook2016}. With this extension, during the recession, any reduction in aggregate consumption below its steady-state level directly reduces aggregate productivity and thus labor income. Hence, any policy stimulating consumption will also boost incomes through this aggregate demand multiplier channel.

%We parametrize the model in two steps.  First, we estimate the extent to which consumers `splurge' when receiving an income shock. We do so using Norwegian data because it offers the best available evidence on the time profile of the marginal propensity to consume (provided by \cite{fagereng_mpc_2021}). Next we move on to the calibration of the full model on US data taking the splurge-factor as given. In the model, consumers are \textit{ex-ante} heterogeneous: The population consists of types that differ according to their level of education (which affects measured facts about permanent income and income dynamics), and their pure time-discount factors, whose distribution is estimated separately for each education group to match the liquid wealth distribution within that group. In addition, agents experience different histories of idiosyncratic income shocks and periods of unemployment, so that within each type there is \textit{ex-post} heterogeneity induced by different shock realizations. 

Our results are intuitive. 

In the economy with no multiplier during recessions, the benefit of a sustained wage-tax cut is small.  One reason there is any benefit at all is that, even for people who have not experienced an unemployment spell, the heightened risk of unemployment during a recession increases the marginal value of current income because it helps them build the extra precautionary reserves they desire to buffer against the extra risk.  A second benefit is that, by the time a person does become unemployed, the temporary tax reduction will have allowed them to accumulate a larger buffer to sustain them during unemployment.  Finally, in a recession there are more people who will have experienced a spell of unemployment, and the larger population of beneficiaries means that the consequences of the prior mechanism will be greater.  But, quantitatively, all of these effects are small.

When a multiplier exists the tax cut has more benefits, especially if the recession continues long enough that most of the spending induced by the tax cut happens while the economy is still in recession (and the multiplier still is in force).  The typical recession, however, ends long before our ``sustained'' wage-tax cut is reversed, and even longer before lower-MPC consumers have spent down most of their extra after-tax income. Accordingly, even in an economy with a multiplier that is powerful during recessions, much of the wage-tax cut's effect on consumption occurs when any multiplier that might have existed in a recession is no longer operative.

Even leaving aside any multiplier effects, the stimulus checks have more value than the wage tax cut, because at least a portion of such checks go to people who are unemployed and therefore have both high MPC's and high marginal utilities (while wage tax cuts by definition go only to persons who are employed and earning wages).  The greatest `bang-for-the-buck' in terms of (unmultiplied) spending comes from the UI insurance extension, because almost \textit{all} of the recipients are in circumstances in which they have a high MPC and a high marginal utility.

<<<<<<< HEAD
And, in contrast to the wage-tax cut, both the UI extension and the stimulus checks concentrate most of the marginal increment to consumption at times when the multiplier (if it exists) is still powerful.  A disadvantage of the UI extension, in terms of `multiplied bang for the buck,' is that more of any extended UI payouts are likely to occur after the recession is over (when by assumption there is no multiplication).  Countering this is the fact that the MPC of UI recipients is higher than that of stimulus check recipients; in the end, our model says that these two forces roughly balance each other, so that the `multiplied bang for the buck' of the two policies is similar.  In the welfare/utility metric, however, there is still considerable marginal value to UI recipients who receive their benefits after the recession is over, so the utility metric boosts the relative value of UI benefits compared to stimulus checks.
=======
We conclude that extended UI benefits should be the first weapon employed from this arsenal.  But a disadvantage is that the total amount of stimulus that can be accomplished with this tool is constrained by the fact that only a limited number of people become unemployed.  If more stimulation is called for than can be accomplished via UI extension, stimulus checks have the advantage that their effects scale almost linearly in the size of the stimulus.  The wage tax cut is also in principle scalable, but it's effects are smaller than those of checks because its recipients have somewhat lower MPCs and marginal utility than check and UI recipients.  In the real world, a tax cut is also likely the least flexible of the three tools:  UI benefits can be further extended, multiple rounds of checks can be sent; but multiple rounds of changes in wage tax rates would likely be administratively and politically more difficult to achieve.
>>>>>>> bdd75cb0b3a84cf519401155286c56d02a20b3b1

We conclude that extended UI benefits should be the first weapon employed from this arsenal, as having a greater welfare/utility benefit than stimulus checks and a similar `bang for the multiplied buck.' But a disadvantage is that the total amount of stimulus that can be accomplished with this tool is constrained by the fact that only a limited number of people become unemployed.  If more stimulation is called for than can be accomplished via UI extension, checks have the advantage that their effects scale almost linearly in the size of the stimulus.  The wage tax cut is also in principle scalable, but it's effects are smaller than those of checks because recipients have lower MPCs and marginal utility than check and UI recipients.  In the real world, a tax cut is also likely the least flexible of the three tools:  UI benefits can be further extended, multiple rounds of checks can be sent; but multiple rounds of changes in wage tax rates would likely be administratively and politically more difficult.

The policies we analyze here are deliberately stylized, and therefore may not match any particular policy actually implemented historically.  But the tools we are using could be easily modified to evaluate a number of other policies.  For example, in the COVID recession, not only was the duration of UI benefits extended, those benefits were supplemented by very substantial extra payments to every UI recipient.  We did not calibrate the model to match this particular policy, but the framework could easily accommodate such an analysis.

\subsection{Related literature}
\label{sec:lit}

This paper is closely related to the empirical literature that aims to estimate the effect of transitory income shocks and stimulus payments. In particular, we focus on \cite{fagereng_mpc_2021} who use Norwegian administrative panel data with sizeable lottery wins to estimate the marginal propensity to consume (MPC) out of transitory income in the quarter it is obtained, as well as the pattern of expenditure in the following quarters. We take their estimates as an input and build a model that is consistent with the patterns they identify. The empirical literature that arose in the aftermath of the Great Recession in 2008 to evaluate the effect of stimulus payments made during the recession, is also closely related. Important examples are \cite{parker2013consumer} and \cite{broda2014economic}. Both of these papers exploit the effectively random timing of the distribution of the payments and identify a substantial consumption response. In our model, consumers do not adjust their labor supply in response to the stimulus policies, which is broadly consistent with the empirical findings in \cite{ganong2022spending}. All these results indicate a substantial MPC that is difficult to reconcile with representative agent models that tend to imply that transitory income shocks are mostly smoothed. 

Thus the paper also relates to the literature presenting models with hetergeneous agents (`HA models') that are built to be consistent with the evidence from micro data discussed above. A key example is \cite{kaplan2014model} who build a model where agents save in both liquid and illiquid assets. In their model they obtain a substantial consumption response to a stimulus payment, since MPCs are high both for constrained, low-wealth households and for households with substantial net worth that is mainly invested in the illiquid asset (the ``wealthy hand-to-mouth''). \cite{carroll2020modeling} present an HA model that is similar in many respects to the one we study. Their focus is on predicting the consumption response to the 2020 U.S. `CARES' act, a policy implemented in the spring of 2020 when a lockdown was in place to limit the spread of the coronavirus. The policy contains both an extension of unemployment benefits and a stimulus check. However, neither of these papers attempts to evaluate and rank different stimulus policies implemented in ``normal'' recessions as we do in this paper. 

In more recent work, \cite{kaplanMPC2022} discuss different mechanisms used in HA models to obtain a high MPC and the tension between that and fitting the wealth distribution. We use one of these mechanisms, ex-ante heterogeneity in discount factors, but also extend the model to include 'splurge' consumption. We obtain a model that delivers both high average MPCs and a distribution of liquid wealth consistent with the data. Therefore, our model does not suffer from what they refer to as the `missing middle' problem. In addition, we not only focus on the initial MPC, but also on the propensity to spend out a windfall for several quarters after it is obtained. 

One of the criteria we use to rank policies is the fiscal multiplier, and our paper therefore relates to the vast literature discussing the size of the multiplier. Our focus is on policies implemented in the aftermath of the Great Recession, a period when monetary policy was essentially fixed at the zero lower bound. We therefore do not consider alternative monetary policy responses to the policies we evaluate, and our work thus relates to papers such as \cite{christiano2011government} and \cite{eggertsson2011fiscal} who argue that fiscal multipliers are higher in such circumstances. \cite{hagedorn2019fiscal} present a HA model with both incomplete markets and nominal rigidities to evaluate the size of the fiscal multiplier in a rich setting. They also find that the multiplier is higher when monetary policy is constrained at the zero lower bound. However, key for their result is not that the nominal rate is stuck at zero, but that it does not respond to the fiscal policy they consider. Unlike us, they focus on government spending and are interested in different options for financing that spending. They do not consider the different policies involving transfers directly to households that we study. \cite{ramey2018government} investigate empirically, using a long historical dataset, whether there is support for the model-based results that fiscal multipliers are higher in certain states. They also focus on government spending and find that the multipliers are generally low. While there is some evidence that multipliers are higher when there is slack in the economy or the ZLB binds, the multipliers they find are still below one in most specifications. In any case, we condition on policies being implemented in a recession, when this literature argues multipliers are higher, but it is not crucial for our purposes whether the multipliers are greater than one or not. We are concerned with relative multipliers, and the multiplier is only one of the two criteria we use to rank the policies we consider. 

We also introduce a measure of welfare to rank policies. Thus, the paper relates to the recent literature on welfare comparisons in HA models. Both \cite{bhandari2021efficiency} and \cite{davila2022welfare} introduce ways of decomposing welfare effects into different terms. In the former case these are aggregate efficiency, redistribution and insurance, while the latter further decomposes the insurance component into intra- and intertemporal components. Even though both of these papers are related, our focus is not on decomposing the welfare effects. We want to use a welfare measure as an additional way of ranking policies, and introduce a measure that abstracts from any incentive for a planner to redistribute in the steady state (or `normal' times).

Finally, a recent related paper is \cite{kekre2022unemployment} who evaluates the impact of extending unemployment insurance in the period from 2008-2014. He finds that this raised aggregate demand and implied a lower unemployment rate than without the extension. However, he does not attempt to compare the stimulus effects of extending unemployment insurance with other policies. 

\onlyinsubfile{% Allows two (optional) supplements to hard-wired \texname.bib bibfile:
% global.bib is a default bibfile that supplies anything missing elsewhere
% Add-Refs.bib is an override bibfile that supplants anything in \texfile.bib or global.bib
\provideboolean{AddRefsExists}
\provideboolean{globalExists}
\provideboolean{BothExist}
\provideboolean{NeitherExists}
\setboolean{BothExist}{true}
\setboolean{NeitherExists}{true}

\IfFileExists{\econtexRoot/Add-Refs.bib}{
  % then
  \typeout{References in Add-Refs.bib will take precedence over those elsewhere}
  \setboolean{AddRefsExists}{true}
  \setboolean{NeitherExists}{false} % Default is true
}{
  % else
  \setboolean{AddRefsExists}{false} % No added refs exist so defaults will be used
  \setboolean{BothExist}{false}     % Default is that Add-Refs and global.bib both exist
}

% Deal with case where global.bib is found by kpsewhich
\IfFileExists{/usr/local/texlive/texmf-local/bibtex/bib/global.bib}{
  % then
  \typeout{References in default global global.bib will be used for items not found elsewhere}
  \setboolean{globalExists}{true}
  \setboolean{NeitherExists}{false}
}{
  % else
  \typeout{Found no global database file}
  \setboolean{globalExists}{false}
  \setboolean{BothExist}{false}
}

\ifthenelse{\boolean{showPageHead}}{ %then
  \clearpairofpagestyles % No header for references pages
  }{} % No head has been set to clear

\ifthenelse{\boolean{BothExist}}{
  % then use both
  \typeout{bibliography{\econtexRoot/Add-Refs,\econtexRoot/\texname,global}}
  \bibliography{\econtexRoot/Add-Refs,\econtexRoot/\texname,global}
  % else both do not exist
}{ % maybe neither does?
  \ifthenelse{\boolean{NeitherExists}}{
    \typeout{bibliography{\texname}}
    \bibliography{\texname}}{
    % no -- at least one exists
    \ifthenelse{\boolean{AddRefsExists}}{
      \typeout{bibliography{\econtexRoot/Add-Refs,\econtexRoot/\texname}}
      \bibliography{\econtexRoot/Add-Refs,\econtexRoot/\texname}}{
      \typeout{bibliography{\econtexRoot/\texname,global}}
      \bibliography{        \econtexRoot/\texname,global}}
  } % end of picking the one that exists
} % end of testing whether neither exists
}

\end{document}
