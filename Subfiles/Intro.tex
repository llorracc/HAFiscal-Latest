\documentclass[../HAFiscal]{subfiles}
\begin{document}
Fiscal policies that aim to boost consumer spending in recessions have been tried repeatedly in many countries in recent years.  The nature of such policies has been quite varied, at least in part because traditional macroeconomic models were unable to provide clear guidance about which policies were likely to be most effective.

But new sources of microeconomic data, such as those from Scandinavian national registries, have recently enabled unprecedentedly fine-grained measurement of the dynamics of different types of consumers' spending patterns in response to income shocks.  Simultaneously, advances in Heterogeneous Agent macro modeling have made it possible to construct structural models capable of matching these spending patterns with a reasonably high degree of fidelity.  This combination of developments makes it possible, really for the first time, to conduct quantitatively credible structural analyses of the likely effectiveness of such policies.

Because spending dynamics in our model reflects the behavior of utility maximizing consumers, we are able to evaluate the policies not only by their effects on aggregate consumption expenditures, but also directly in terms of the impact on consumers' utility.  The principal difference between the two metrics is that is that the utility-metric evaluation further increases the already considerable advantage that the UI extension exhibited in the consumption-boosting metric; the benefits of the UI extension are greater since the payments are specifically directed to a set of consumers who have high marginal utility.

Our model builds upon a now standard buffer-stock saving model of consumption to which we add features that we believe are important to capture the effects of fiscal stimulus policies. The most important of these is that consumers spend a fixed fraction of their labor income each period, which we call the `splurge' factor. This spending occurs regardless of their current wealth and fits with the empirical evidence that even high liquid-wealth households have high initial MPCs. By contrast, in a standard buffer-stock model, high-wealth households smooth their consumption through transitory shocks and exhibit low MPCs. We use the model to aggregate consumer utility into a social welfare function. Because low-income consumers have high marginal utility, a standard aggregated welfare function would favor re-distributive policies even in the absence of a recession. To avoid weighting our analysis toward re-distributive policies, we normalize our social welfare criteria such that each policy, implemented in non-recessionary times, has zero welfare benefit to the social planner. 

Recessions are unexpected (they are `MIT shocks') and double the unemployment rate and the average length of unemployment spells. The end of the recession occurs as a Bernoulli process calibrated for an average length of recession of six quarters, leading to a return of the unemployment rate to normal levels over time. In an extension to the model, we allow for an aggregate demand multiplier effect during the recession, following the method introduced by \cite{kmpHandbook2016}. With this extension, during the recession, any reduction in aggregate consumption below its steady-state level directly reduces aggregate productivity and thus labor income. Hence, any policy stimulating consumption will also boost incomes through this aggregate demand multiplier channel.

- [ Briefly elaborate on the ways in which we have calibrated the model to match ``dynamics of the MPC'' (splurge, matching liquid asset distribution, etc).  Mention that we are mixing-matching US and Norwegian data and briefly defend, but say that details and a more extended justification will follow. - HT and/or IF]
We parametrize the model in two steps.  First, we estimate the extent to which consumers `splurge' when receiving an income shock. We do so using Norwegian data because it offers the best available evidence on the time profile of the marginal propensity to consume (provided by \citet{fagereng_mpc_2021}). Next we move on to the calibration of the full model on US data taking the splurge-factor as given. In the model, consumers are \textit{ex-ante} heterogeneous: The population consists of types that differ according to their level of education (which affects measured facts about permanent income and income dynamics), and their pure time-discount factors, whose distribution is estimated separately for each education group to match the liquid wealth distribution within that group. In addition, agents experience different histories of idiosyncratic income shocks and periods of unemployment, so that within each type there is \textit{ex-post} heterogeneity induced by different shock realizations. 

Our results are intuitive.

In the economy with no multiplier during recessions, the benefit of a sustained wage tax cut is small.  One reason there is any benefit at all is that, even for people who have not experienced an unemployment spell, the heightened risk of unemployment during a recession increases the marginal value of income because it helps them build the extra precautionary reserves they desire because of the extra risk.  A second benefit is that, by the time a person does become unemployed, the temporary tax reduction will have allowed them to accumulate a larger buffer stock to sustain them during unemployment.  Finally, in a recession there are more people who will have experienced a spell of unemployment, and the larger population of beneficiaries means that the consequences of the two prior mechanisms will be greater.  But, quantitatively, all of these effects are small.

When a multiplier exists, the tax cut has more benefits, especially if the recession continues long enough that most of the spending induced by the tax cut happens while the economy is still in recession (and therefore the multiplier still is in force).  The typical recession, however, ends long before our ``sustained'' tax cut is reversed, so even in an economy with a multiplier that is powerful during recessions, much of the tax cut's effect on consumption occurs when any multiplier that might have existed in a recession is no longer operative.

In contrast to the tax cut, both the UI extension and the stimulus checks concentrate most of the marginal increment to consumption at times when the multiplier (if it exists) is still powerful.  Even leaving aside any multiplier effects, the stimulus checks have more value than the wage tax cut, because at least a portion of them go to people who are unemployed and therefore have both high MPC's and high marginal utilities (while wage tax cuts by definition go only to persons who are employed and earning wages).  But the greater bang-for-the-buck of the UI extension reflects the fact that \textit{all} of the recipients are in circumstances in which they have a high MPC and a high marginal utility.

We conclude that extended UI benefits should be the first weapon employed from this arsenal.  But a disadvantage is that the total amount of stimulus that can be accomplished with this tool is constrained by the fact that only a limited number of people become unemployed.  If more stimulation is called for than can be accomplished via UI extension, checks have the advantage that their effects scale almost linearly in the size of the stimulus.  The wage tax cut is also in principle scalable, but it's effects are smaller than those of checks because its recipients have considerably lower MPCs and marginal utility than check and UI recipients.  In the real world, a tax cut is also likely the least flexible of the three tools:  UI benefits can be further extended, multiple rounds of checks can be sent; but multiple rounds of changes in wage tax rates would likely be administratively and politically more difficult to achieve.

The tools we are using could be reasonably easily modified to evaluate a number of other policies.  For example, in the COVID recession, not only was the duration of UI benefits extended, those benefits were supplemented by very substantial extra payments to every UI recipient.  We did not calibrate the model to match this particular policy, but the framework could easily accommodate such an analysis.

\end{document}
