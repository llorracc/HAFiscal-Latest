\documentclass[../HAFiscal]{subfiles}
\begin{document}

\section{Conclusion}

For many years leading up to the Great Recession, a widely held view among macroeconomists was that countercyclical policy should be left to central banks, because fiscal policy responses were unpredictable in their timing, their content, and their effects.  Nevertheless, even during this period, fiscal policy responses to recessions were repeatedly tried -- perhaps because the macroeconomists' advice to policymakers -- ``don't just do something -- stand there'' -- is not politically tenable.

This paper demonstrates that macroeconomic modeling has finally advanced to the point where we can make reasonably credible assessments of the effects of alternative policies of the kinds that have been tried.  The key developments have been the advent of national registry datasets that can measure crucial microeconomic phenomena, and the development of tools of heterogeneous agent macroeconomic modeling that can match those micro facts and glean their macroeconomic implications.

We examine three fiscal policy experiments that have actually been implemented in the past: an extension of UI benefits, a stimulus check and a tax cut on labor income.  Our model suggests that the extension of UI benefits is a clear ``bang for the buck'' winner.  Not only is it the policy that yields the greatest spending boost during while the recession is ongoing (when multipliers are likely to be strongest), it also leads to the greatest welfare gains. The chief drawback of the UI extension is that its size is limited by the fact that a relatively small share of the population is affected by it. In contrast, stimulus checks are easily scaleable while exhibiting only slightly less recession-period stimulus (in a typical recession). However, since some of the stimulus checks flow to well-off consumers, it does worse than UI extensions when we evaluate welfare consequences. Finally, the tax cut is the least effective both in terms of the multiplier and welfare impact since it only targets employed consumers, and for a typical recession more of its payouts are likely to occur after the recessionary period (when multipliers may exist) has ended.

The tools we are using could be reasonably easily modified to evaluate a number of other policies.  For example, in the COVID recession, not only was the duration of UI benefits extended, those benefits were supplemented by very substantial extra payments to every UI recipient.  We did not calibrate the model to match this particular policy, but the framework could easily accommodate such an analysis.


\end{document}