\documentclass[../HAFiscal]{subfiles}
\begin{document}
	
\section{Conclusion}

In this paper we have investigated the effectiveness of three fiscal stimulus polices: an extension of UI benefits, a stimulus check and a tax cut on labor income. Our model suggests that the extension of UI benefits is a clear "bang for the buck winner" out of the three stimulus policies considered. Not only is it the policy with the highest multiplier, it also leads to the highest welfare gains, even though our welfare measure is neutral with respect to redistribution of income. The UI extension policy is, however, limited in its size since a relatively small share of the population is affected by it. In contrast, the stimulus check is easily scaleable while exhibiting an equally high multiplier value as the UI extension. However, since parts of the stimulus check flow to well-off consumers, the policy does less well in terms of our welfare measure. Finally, the tax cut is the least effective both in terms of the multiplier and welfare impact since it only targets employed consumers.

The tools we are using could be reasonably easily modified to evaluate a number of other policies.  For example, in the COVID recession, not only was the duration of UI benefits extended, those benefits were supplemented by very substantial extra payments to every UI recipient.  We did not calibrate the model to match this particular policy, but the framework could easily accommodate such an analysis.


\end{document}	