\documentclass[11pt]{article}
\usepackage[utf8]{inputenc} 
\usepackage[T1]{fontenc}
\usepackage{sectsty}
\usepackage{graphicx}
\usepackage{amsmath}
\usepackage{booktabs}
\usepackage{placeins}

% Margins
\topmargin=-0.45in
\evensidemargin=0in
\oddsidemargin=0in
\textwidth=6.5in
\textheight=9.0in
\headsep=0.25in

\title{ HAFiscal project paper outline}
\author{Christopher Carroll, Edmund Crawley, Ivan Frankovic, Håkon Tretvoll}
\date{\today}

\begin{document}
	\maketitle
	
	
	\section{Introduction}
	
	\section{Model}
	
	\section{Estimation and calibration}
	
	\section{Fiscal policy simulations}
	
	We consider the following fiscal policy experiments
	
	\begin{itemize}
		\item Payroll tax cut: Employed individuals benefit from a 2 percentage points lower payroll tax cut. The tax cut is unanticipated and usually lasts for 8 quarters. However, there is a 50\% chance, that the policy is extended by another 8 quarters if the recession is still ongoing in the 8th quarter of the payroll tax cut. 
		\item Unemployment insurance extension: The duration of the unemployment insurance is doubled from 2 to 4 quarters. Agents, that are unemployed when the policy is implemented thus receive up to 4 quarters of unemployment insurance. The policy is unanticipated and active only for one quarter.
		\item Stimulus check: Each individual, independent of employment status, receives an unanticipated payment of \$1200 in one quarter. However, the check is only paid out fully to individuals with a permanent yearly income smaller than 100,000 and not at all to those with a income greater than 150,000. Those within the two thressholds receive a share of the full stimulus check amount proportionate to their position within threshholds.\footnote{For this income group, the check amount is given by $\$1200 (1-\frac{Income-100,000}{50,000})$. For example, an individual with a permanent yearly income of 110,000 receives 80\% of the stimulus, i.e. \$960.}
	\end{itemize}
	
	\subsection{Impulse responses}
	
	
	\begin{figure}
		\centering
		\includegraphics[width=0.8\linewidth]{../Code/HA-Models/FromPandemicCode/Figures/recession_taxcut_relrecession}
		\caption{Impulse responses of aggregate income and consumption to a pay roll tax cut during a recesssion lasting eight quarters with and without aggregate demand effects}
		\label{fig:recessiontaxcutrelrecession}
	\end{figure}
	
	\begin{figure}
		\centering
		\includegraphics[width=0.8\linewidth]{../Code/HA-Models/FromPandemicCode/Figures/recession_UI_relrecession}
		\caption{Impulse responses of aggregate income and consumption to a UI extension during a recesssion with and without aggregate demand effects}
		\label{fig:recessionuirelrecession}
	\end{figure}
	
	\begin{figure}
		\centering
		\includegraphics[width=0.8\linewidth]{../Code/HA-Models/FromPandemicCode/Figures/recession_Check_relrecession}
		\caption{Impulse responses of aggregate income and consumption to a stimulus check during a recesssion with and without aggregate demand effects}
		\label{fig:recessioncheckrelrecession}
	\end{figure}

	\FloatBarrier
	\subsection{Multipliers}
	
	Definitions:
	\begin{itemize}
		\item The \textit{net present value (NPV)} of a variable X at horizon t is given by
		\begin{equation}
			NPV(t,X) = \sum_{s=0}^{t} \left( \prod_{i=1}^{s} \frac{1}{R_i} \right) X_s
		\end{equation}
		\item The \textit{cummulative multiplier (CM)} of a policy is given by
		\begin{equation}
			CM(t) = \frac{NPV(t,\Delta C)}{NPV (T_{max},\Delta G)}
		\end{equation}
		where $\Delta C$ is the additional aggregate consumption spending in the policy scenario relative to the baseline and $\Delta G$ is the government expenditures caused by the policy.
	\end{itemize}
	
	\begin{table} 
		\center
		\input ../Code/HA-Models/FromPandemicCode/Tables/Multiplier.tex
		\caption{Multipliers as well as the share of the policy ocurring during the recession for the three policies considered}
		\label{tab:Multiplier}
	\end{table}
	
	\begin{figure}
		\centering
		\includegraphics[width=0.8\linewidth]{../Code/HA-Models/FromPandemicCode/Figures/Cummulative_multipliers}
		\caption{Cummulative Multiplier as a function of the horizon in quarters for the three policies considered. Policies are implemented during a recession with AD effects active}
		\label{fig:cummulativemultipliers}
	\end{figure}
	
	\FloatBarrier
	\section{Welfare analysis}
	
	We want to convert welfare units to consumption units. A proportional increase in every agents' consumption in the baseline by fraction $x$, in welfare, is equal to:
	\begin{align}
	x\frac{1}{N}\sum_{i=1}^{N} \sum_{t=0}^{\infty} D^t c_{it,\text{base}} u'(c_{it,\text{base}})
	\end{align}
	where $c_{it}$ is consumption (including the splurge) of agent $i$ at time $t$ and $D$ is the social planner's discount rate. $N$ is the number of agents.
	
	
	The cost of such an increase is
	\begin{align}
	x\frac{1}{N}\sum_{i=1}^{N} \sum_{t=0}^{\infty} R^{-t} c_{it,\text{base}}
	\end{align}
	Define
	\begin{align}
	\mathcal{W}^c =\frac{1}{N}\sum_{i=1}^{N} \sum_{t=0}^{\infty} D^t c_{it,\text{base}} u'(c_{it,\text{base}}) \\
	\mathcal{P}^c = \frac{1}{N}\sum_{i=1}^{N} \sum_{t=0}^{\infty} R^{-t} c_{it,\text{base}}
	\end{align}
	Aside - with log utility, $\mathcal{W}^c =\frac{1}{N}\sum_{i=1}^{N} \sum_{t=0}^{\infty} D^t = \frac{1}{1-D}$
	
	We will assume that a government expenditure of size $F$ with welfare benefit $\mathcal{W}$ will be funded by a proportional consumption tax of size $\frac{F}{\mathcal{P}^c}$ resulting in a welfare loss of $\frac{F}{\mathcal{P}^c}\mathcal{W}^c$. The overall welfare benefit will be equivalent to consumption units:
	\begin{align}
	\mathcal{C} = \frac{\mathcal{W}}{\mathcal{W}^c} - \frac{F}{\mathcal{P}^c}
	\end{align}
	There is also an `unseen' cost to the government policy exactly equal to implementing the policy in normal times.
	
	Define welfare of a policy as:
	\begin{align}
	\mathcal{W}(\text{policy},AD,Rec) = \frac{1}{N}\sum_{i=1}^{N} \sum_{t=0}^{\infty} D^t u(c_{it,\text{policy},AD,Rec})
	\end{align}
	
	So the consumption equivalent of a policy implemented in recession is:
	\begin{align}
	\mathcal{C}(\text{policy},AD,Rec) &= \bigg(\frac{\mathcal{W}(\text{policy},AD,Rec)-\mathcal{W}(AD,Rec)}{\mathcal{W}^c} - \frac{PV(\text{policy},Rec)}{\mathcal{P}^c} \bigg)\\ \nonumber
	& \qquad -
	\bigg(\frac{\mathcal{W}(\text{policy}) - \mathcal{W}(\text{base})}{\mathcal{W}^c} - \frac{PV(\text{policy})}{\mathcal{P}^c} \bigg)
	\end{align}
	
	\begin{table} 
		\center
		\input ../Code/HA-Models/FromPandemicCode/Tables/welfare4.tex
		\caption{Consumption Equivalent Welfare Gains in Basis Points }
		\label{welfare}
	\end{table}
	
	Table \ref{welfare} shows results for this method. Note that the policy expenditures of each policy have been equalized.
	
	\section{Conclusion}
	
	\appendix
	\section{Appendix section example}
	
	
	
\end{document}