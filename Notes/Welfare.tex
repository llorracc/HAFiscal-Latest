\documentclass[11pt]{article}

\usepackage{sectsty}
\usepackage{graphicx}
\usepackage{amsmath}
\usepackage{booktabs}

% Margins
\topmargin=-0.45in
\evensidemargin=0in
\oddsidemargin=0in
\textwidth=6.5in
\textheight=9.0in
\headsep=0.25in

\title{ Welfare}
\author{ Edmund Crawley }
\date{\today}

\begin{document}
	\maketitle
	All results here are based on a simulation out to 40 quarters, so miss welfare benefits beyond that.
		
	\section{Method 1: Subtract Welfare Gains in Normal Times}
	The social planner has the same weight on the welfare of each agent and discounts according to some discount factor (results here are discounted at the risk free weight).
	
	For a simulation of $N$ agents define $\mathcal{W}$ as the social planners welfare:
	\begin{align}
		\mathcal{W} = \frac{1}{N}\sum_{i=1}^{N} \sum_{t=0}^{\infty} D^t u(c_{it})
	\end{align}
	where $c_{it}$ is consumption (including the splurge) of agent $i$ at time $t$ and $D$ is the social planner's discount rate.
	
	We calculate the social planner's welfare under a number of different simulations:
	\begin{itemize} 
		\item  $\mathcal{W}(\text{Base})$ is welfare in normal, steady state times
		\item  $\mathcal{W}(\text{Check})$, $\mathcal{W}(\text{UI})$, $\mathcal{W}(\text{TaxCut})$ are welfare in normal, steady state times with one of the three policies implemented
		\item    $\mathcal{W}(\text{Rec})$ is welfare in a recession, no policy or AD effects
		\item  $\mathcal{W}(\text{Rec,Check})$, $\mathcal{W}(\text{Rec,UI})$, $\mathcal{W}(\text{Rec,TaxCut})$ are welfare in a recession with one of the policies but no AD effects
		\item    $\mathcal{W}(\text{Rec,AD})$ is welfare in a recession with AD effects but no policy
		\item  $\mathcal{W}(\text{Rec,AD,Check})$, $\mathcal{W}(\text{Rec,AD,UI})$, $\mathcal{W}(\text{Rec,AD,TaxCut})$ are welfare in a recession with AD effects and one of the policies
	\end{itemize}
	We also define $PV(\text{policy})$, $PV(Rec,\text{policy})$ and $PV(Rec, AD,\text{policy})$ in a similar way as the present value of the fiscal payments made to pay for each policy under normal times, recession without AD effects and a recession with AD effect.
	
	We are interested in the welfare gain of the policy, making the assumption that there is a welfare loss associated with the policy equal to the welfare gain of the policy in normal times.
	
	Define welfare gain per dollar spent, $\mathcal{G}(\text{policy})$, to be the welfare gain of the policy in a recession minus the welfare gain of the policy in normal times, divided by the cost of the policy in a recession:
	\begin{align}
		\mathcal{G}(\text{policy}) = \frac{\big(\mathcal{W}(\text{Rec,policy}) - \mathcal{W}(\text{Rec})\big)  - \big(\mathcal{W}(\text{policy}) - \mathcal{W}(\text{Base})\big) }{PV(Rec,\text{policy})}
	\end{align}
	Similarly for an AD economy we define:
	\begin{align}
	\mathcal{G}(AD,\text{policy}) = \frac{\big(\mathcal{W}(\text{Rec,AD,policy}) - \mathcal{W}(\text{Rec,AD})\big)  - \big(\mathcal{W}(\text{policy}) - \mathcal{W}(\text{Base})\big) }{PV(Rec,\text{policy})}
	\end{align}
	Note in this definition we divide by the cost of the policy in a recession without AD effects as the `cost' of a payroll tax cut goes up as incomes rise, even though the govt's budget will improve.
	
	The results of this calculation are shown in table \ref{welfare1}. An initial reading of the results would suggest the UI gains are much larger than either checks or Tax Cuts.
	\begin{table} 
		\center
		\input ../Code/HA-Models/FromPandemicCode/Tables/welfare1.tex
		\caption{Welfare gains}
		\label{welfare1}
	\end{table}
	There is, however, one big problem with this method. If $PV(Rec,\text{policy})$ is different to $PV(\text{policy})$, then the difference in welfare between the policy in a recession and the policy in normal times does not account for this increase in policy cost. This is a large part of the reason why the UI policy comes out looking so good - it has very little effect in normal times because hardly anyone needs extended UI, so $\mathcal{W}(\text{UI}) - \mathcal{W}(\text{Base})$ is small relative to the cost of the policy in recession. This seems unfair on the other polices - the tax cut is actually \textit{more} expensive in normal times.
	
	An alternative would be to take away ``welfare per dollar'' in normal times. That is define $\mathcal{G}2$ as:
	\begin{align}
	\mathcal{G}2(\text{policy}) = \frac{\mathcal{W}(\text{Rec,policy}) - \mathcal{W}(\text{Rec})}{PV(Rec,\text{policy})}  - \frac{(\mathcal{W}(\text{policy}) - \mathcal{W}(\text{Base}) }{PV(\text{policy})}
\end{align}
and 
	\begin{align}
	\mathcal{G}2(AD,\text{policy}) = \frac{\mathcal{W}(\text{Rec,AD,policy}) - \mathcal{W}(\text{Rec,AD})}{PV(Rec,\text{policy})}  - \frac{(\mathcal{W}(\text{policy}) - \mathcal{W}(\text{Base}) }{PV(\text{policy})}
\end{align}	
The results of this method are reported in table \ref{welfare2}. Now the UI has negative welfare per dollar: households who end up needing extended UI in normal times have already run down their wealth, while in a recession there is more precautionary saving so the welfare gain per dollar is actually higher in normal times than in a recession. Note the check policy welfare is unchanged and the tax cut policy slightly improved.
	\begin{table} 
	\center
	\input ../Code/HA-Models/FromPandemicCode/Tables/welfare2.tex
	\caption{Welfare gains, alternative method}
	\label{welfare2}
\end{table}

	\section{Method 2: Individual Weights to neutralize redistribution in normal times}
	An alternative method is to set the social planners weights so that, in normal times, they have no incentive to redistribute consumption across people or time. That is the social planner has weights $w_{i,t}$ for each agent $i$ at time $t$, with risk free rate $R$, such that:
	\begin{align}
		w_{it} = \frac{1}{R^t u'(c_{it,\text{Base}})}
	\end{align}
	The social planner's welfare is then
	\begin{align}
	\mathcal{W}(Rec,AD,\text{policy}) = \frac{1}{N}\sum_{i=1}^{N} \sum_{t=0}^{\infty} w_{it} u(c_{it,Rec,AD,\text{Policy}})
	\end{align}
	By design, a policy that adds $\varepsilon$ of extra consumption will have a welfare gain of one per one per dollar spent in normal times. This one-for-one welfare gain per dollar will go down as the size of the policy increases, due to concavity.
	
	Define
	\begin{align}
	\mathcal{G}3(\text{policy}) &= \frac{\mathcal{W}(\text{policy}) - \mathcal{W}(\text{Base})}{PV(\text{policy})} \\
	\mathcal{G}3(Rec,\text{policy}) &= \frac{\mathcal{W}(\text{Rec,policy}) - \mathcal{W}(\text{Rec})}{PV(Rec,\text{policy})} \\
	\mathcal{G}3(Rec,AD,\text{policy}) &= \frac{\mathcal{W}(\text{Rec,AD,policy}) - \mathcal{W}(\text{Rec,AD})}{PV(Rec,\text{policy})}
	\end{align}
	
	The top row of table \ref{welfare3} shows the welfare gain per dollar of each policy in normal times. The check and tax cut numbers are below one primarily because the implementation only measures consumption out for 40 quarters currently. The UI policy is below one because each recipient of extended UI moves significantly along their consumption function so that the concavity kicks in.
	
	The second two rows of table \ref{welfare3} show the welfare benefit per dollar spent when the policy occurs during a recession, and a recession with AD. The stimulus check is slightly more effective during a recession because those who have become unemployed will get a greater welfare gain from it. The extended UI policy is far more cost effective because it gets sent to a large number of agents who are now unemployed for a long time who would have been employed (and therefore high social planner weights). I'm not sure why the tax cut welfare goes up - my expectation is that it should go down slightly.
	
	The AD row shows the AD effects multiply the welfare gains for each policy.
	
	\paragraph{How could we justify these weights?}
	We can argue that in ordinary times, extended UI benefits will have incentive costs or could just go to those who are not really looking for work. These do not apply to the extra individuals who have become unemployed due to the recession and would be working otherwise.
	\begin{table} 
	\center
	\input ../Code/HA-Models/FromPandemicCode/Tables/welfare3.tex
	\caption{Welfare gains}
	\label{welfare3}
	\end{table}
	
\section{Method 3: Subtract Consumption tax to account for the cost of the policy}

We want to convert welfare units to consumption units. A proportional increase in every agents' consumption in the baseline by fraction $x$, in welfare, is equal to:
	\begin{align}
	 x\frac{1}{N}\sum_{i=1}^{N} \sum_{t=0}^{\infty} D^t c_{it,\text{base}} u'(c_{it,\text{base}})
	\end{align}
The cost of such an increase is
	\begin{align}
		x\frac{1}{N}\sum_{i=1}^{N} \sum_{t=0}^{\infty} R^{-t} c_{it,\text{base}}
	\end{align}
Define
	\begin{align}
	\mathcal{W}^c =\frac{1}{N}\sum_{i=1}^{N} \sum_{t=0}^{\infty} D^t c_{it,\text{base}} u'(c_{it,\text{base}}) \\
	\mathcal{P}^c = \frac{1}{N}\sum_{i=1}^{N} \sum_{t=0}^{\infty} R^{-t} c_{it,\text{base}}
\end{align}
Aside - with log utility, $\mathcal{W}^c =\frac{1}{N}\sum_{i=1}^{N} \sum_{t=0}^{\infty} D^t = \frac{1}{1-D}$

We will assume that a government expenditure of size $F$ with welfare benefit $\mathcal{W}$ will be funded by a proportional consumption tax of size $\frac{F}{\mathcal{P}^c}$ resulting in a welfare loss of $\frac{F}{\mathcal{P}^c}\mathcal{W}^c$. The overall welfare benefit will be equivalent to consumption units:
	\begin{align}
	\mathcal{C} = \frac{\mathcal{W}}{\mathcal{W}^c} - \frac{F}{\mathcal{P}^c}
	\end{align}
There is also an `unseen' cost to the government policy exactly equal to implementing the policy in normal times.

Define welfare of a policy as:
	\begin{align}
	\mathcal{W}(\text{policy},AD,Rec) = \frac{1}{N}\sum_{i=1}^{N} \sum_{t=0}^{\infty} D^t u(c_{it,\text{policy},AD,Rec})
\end{align}

So the consumption equivalent of a policy implemented in recession is:
	\begin{align}
	\mathcal{C}(\text{policy},AD,Rec) &= \bigg(\frac{\mathcal{W}(\text{policy},AD,Rec)-\mathcal{W}(AD,Rec)}{\mathcal{W}^c} - \frac{PV(\text{policy},Rec)}{\mathcal{P}^c} \bigg)\\ \nonumber
	& \qquad -
	\bigg(\frac{\mathcal{W}(\text{policy}) - \mathcal{W}(\text{base})}{\mathcal{W}^c} - \frac{PV(\text{policy})}{\mathcal{P}^c} \bigg)
	\end{align}

	\begin{table} 
	\center
	\input ../Code/HA-Models/FromPandemicCode/Tables/welfare4.tex
	\caption{Consumption Equivalent Welfare Gains in Basis Points }
	\label{welfare4}
\end{table}

Table \ref{welfare4} shows initial results for this method, without sizing the policies equivalently. For reference, the PV of each policy is: Check (8,000), UI (950), and Tax Cut (45,000)
	
\end{document}